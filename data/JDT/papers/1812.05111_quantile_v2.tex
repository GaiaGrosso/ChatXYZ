\documentclass[aps,prl,twocolumn,preprintnumbers,superscriptaddress,showpacs,floatfix, nofootinbib]{revtex4-1}

\usepackage[caption=false]{subfig}
\usepackage{hyperref}
\usepackage{color}
\usepackage{xcolor}
\usepackage{graphicx}
\usepackage{amsmath}	
\usepackage{siunitx}
\graphicspath{
  {./},
}

\usepackage{amstext}
\usepackage{array}

\usepackage[normalem]{ulem}

\newcommand{\Z}{\rm Z}
\newcommand{\pTZ}{p_T^{\rm Z}}
\newcommand{\pTin}{p_T^{\rm in}}
\newcommand{\pTMC}{p_T^{\rm MC}}
\newcommand{\pTHI}{p_T^{\rm AA}}
\newcommand{\pTvac}{p_T^{\rm pp}}
\newcommand{\pTq}{p_T^{\rm quant}}
\newcommand{\jb}[1]{\textcolor{red}{\bf (#1 --jb)}}
\newcommand{\jbG}[1]{\textcolor{blue}{\bf (#1 --j4g)}}
\newcommand{\jdt}[1]{\textcolor{olive}{\bf (#1 --jdt)}}
\newcommand{\gm}[1]{\textcolor{orange}{\bf #1 (--gm)}}



\newcommand{\Eq}[1]{Eq.~\eqref{#1}}
\newcommand{\Fig}[1]{Fig.~\ref{#1}}
\newcommand{\Ref}[1]{Ref.~\cite{#1}}
\newcommand{\df}{\text{d}}

\begin{document} 

\title{Sorting out quenched jets}

\author{Jasmine Brewer}
\email{jtbrewer@mit.edu}
\affiliation{Center for Theoretical Physics, Massachusetts Institute of Technology, Cambridge, MA 02139, USA}

\author{Jos\'e Guilherme Milhano}
\email{guilherme.milhano@tecnico.ulisboa.pt}
\affiliation{LIP, Av.\ Prof.\ Gama Pinto, 2, P-1649-003 Lisboa, Portugal}
\affiliation{Instituto Superior T\'ecnico (IST), Universidade de Lisboa, Av.\ Rovisco Pais 1, 1049-001, Lisbon, Portugal}

\author{Jesse Thaler}
\email{jthaler@mit.edu}
\affiliation{Center for Theoretical Physics, Massachusetts Institute of Technology, Cambridge, MA 02139, USA}
\affiliation{Department of Physics, Harvard University, 17 Oxford Street, Cambridge, MA 02138, USA}

\begin{abstract}
%
We introduce a new ``quantile'' analysis strategy to study the modification of jets as they traverse through a droplet of quark-gluon plasma.
%
To date, most jet modification studies have been based on comparing the jet properties measured in heavy-ion collisions to a proton-proton baseline at the same reconstructed jet transverse momentum ($p_T$).
%
It is well known, however, that the quenching of jets from their interaction with the medium leads to a migration of jets from higher to lower $p_T$, making it challenging to directly infer the degree and mechanism of jet energy loss.
%
Our proposed quantile matching procedure is inspired by (but not reliant on) the approximate monotonicity of energy loss in the jet $p_T$.
%
In this strategy, jets in heavy-ion collisions ordered by $p_T$ are viewed as modified versions of the same number of highest-energy jets in proton-proton collisions, and the fractional energy loss as a function of jet $p_T$ is a natural observable ($Q_{\rm AA}$).
%
Furthermore, despite non-monotonic fluctuations in the energy loss, we use an event generator to validate the strong correlation between the $p_T$ of the parton that initiates a heavy-ion jet and the $p_T$ of the vacuum jet which corresponds to it via the quantile procedure ($\pTq$).
%
We demonstrate that this strategy both provides a complementary way to study jet modification and mitigates the effect of $p_T$ migration in heavy-ion collisions.

\end{abstract}

\preprint{MIT-CTP/5089}
\maketitle

The deconfined phase of QCD matter, the quark-gluon plasma, was first discovered in collisions of heavy nuclei at the Relativistic Heavy Ion Collider \cite{Adler:2001nb,Arsene:2004fa,Back:2004je,Adcox:2004mh,Adams:2005dq} and confirmed at the Large Hadron Collider \cite{Aamodt:2010pa,Aad:2010bu,Chatrchyan:2011sx}.
%
As in high-energy proton-proton collisions, heavy-ion collisions produce collimated sprays of particles, called jets, from highly energetic scatterings of quarks and gluons.
%
The observation of ``jet quenching''---a strong suppression and modification of jets in heavy-ion collisions \cite{Aad:2010bu,Chatrchyan:2011sx,Adam:2015ewa}---ushered in a new era of studying the properties of the quark-gluon plasma by measuring its effect on jets \cite{Appel:1985dq,Blaizot:1986ma,Gyulassy:1990ye,Wang:1991xy,Chatrchyan:2012nia,Chatrchyan:2012gt,Chatrchyan:2013exa,Chatrchyan:2014ava,Khachatryan:2015lha,Aaboud:2017eww,Acharya:2017goa,Aaboud:2018twu,Aaboud:2018hpb,Acharya:2018uvf}. 


A central issue in interpreting jet quenching measurements is that medium-induced modifications necessarily affect how jets are identified experimentally.
%
Current methods compare proton-proton and heavy-ion jets of the same final (reconstructed) transverse momentum $p_T$ and, as such, inevitably suffer from significant biases from the migration of jets from higher to lower $p_T$ due to medium-induced energy loss (see \cite{CasalderreySolana:2007pr,dEnterria:2009xfs}).
%
While these methods have been very successful in qualitatively demonstrating the phenomena of jet quenching, quantitive studies often necessitate interpreting the data through theoretical models which include migration effects.
%
Ideally, one would like to isolate samples of jets in proton-proton and heavy-ion collisions which were statistically equivalent when they were produced, differing only by the effects of the plasma.

\begin{figure*}
\subfloat[\label{fig:quant_a}]{%
  \includegraphics[width=.5\linewidth]{figure1a.pdf}%
}
\subfloat[\label{fig:quant_b}]{%
  \includegraphics[width=.5\linewidth]{figure1b.pdf}%
}
\caption{\label{fig:quant}
Illustration comparing the ratio and quantile procedures.
(a)
%
The inclusive jet $p_T$ spectra measured by CMS \cite{Khachatryan:2016jfl}, for a jet radius of $R=0.4$.
%
The standard jet ratio $R_{\rm AA}$ (blue) compares heavy-ion and proton-proton jet cross-sections vertically at the same reconstructed jet $p_T$.
%
(b)
%
The jet $p_T$ cumulative cross-sections extracted from \textsc{Jewel} \cite{Zapp:2013vla,KunnawalkamElayavalli:2016ttl}.
%
The quantile procedure $Q_{\rm AA}$ (red) compares heavy-ion and proton-proton jet $p_T$ thresholds horizontally at the same cumulative cross-section.
%
From this, one can map each $\pTHI$ (base of red arrows) into the $p_T$ of proton-proton jets in the same quantile, $\pTq$ (tip of red arrows).
%
For completeness, we also show the pseudo-quantile $\widetilde{Q}_{\rm AA}$ (orange, with corresponding $\widetilde{p}_T^{\rm quant}$) defined on the cross-section and pseudo-ratio $\widetilde{R}_{\rm AA}$ (purple) defined on the cumulative cross-section.
%
Though we will not explore the use of $\widetilde{Q}_{\rm AA}$ or $\widetilde{R}_{\rm AA}$ in the present study, we note that in \textsc{Jewel}, the values of $\pTq$ and $\widetilde{p}_T^{\rm quant}$ differ by only a few percent.
}
\label{fig:quantile-cartoon}
\end{figure*}

In this Letter, we propose a novel data-driven strategy for comparing heavy-ion ($\rm AA$) jet measurements to proton-proton ($\rm pp$) baselines which mitigates, to a large extent, the effect of $p_T$ migration.
%
The famous jet ratio $R_{\rm AA}$ compares the effective cross-section for jets in proton-proton and heavy-ion collisions with the same reconstructed $p_T$:
%
\begin{equation}
	R_{\rm AA} = \left. \frac{\sigma^{\rm eff}_{\rm AA}}{\sigma^{\rm eff}_{\rm pp}} \right|_{p_T},
\end{equation} 
%
as illustrated in blue in \Fig{fig:quant_a}.
%
Here, we introduce a ``quantile'' procedure, which divides jet samples sorted by $p_T$ into quantiles of equal probability.
%
Our new proposed observable for heavy-ion collisions is the $p_T$ ratio between heavy-ion and proton-proton jets in the same quantile:
%
\begin{equation}
\label{eq:QAAdef}
	Q_{\rm AA} = \left. \frac{\pTHI}{\pTvac} \right|_{\Sigma^{\rm eff}},
\end{equation} 
%
as illustrated in red in \Fig{fig:quant_b}, where $1-Q_{\rm AA}$ is a proxy for the average fractional jet energy loss.
%
($Q_{\rm AA}$ is not related to $Q_{\rm pA}$ used by ALICE \cite{Toia:2014wia}).
%
Although $R_{\rm AA}$ can be obtained from $Q_{\rm AA}$ if the proton-proton jet spectrum is known, we will see that the physics interpretation of $R_{\rm AA}$ and $Q_{\rm AA}$ can be quite different.
%
\Fig{fig:quant_a} additionally shows the pseudo-quantile $\widetilde{Q}_{\rm AA}$, which is related to the observable $S_{\rm loss}$ introduced by PHENIX for single hadrons~\cite{Adler:2006bw,Adare:2012wg,Adare:2015cua}.



To give an intuitive understanding of \Eq{eq:QAAdef}, consider a simplified scenario where medium-induced energy loss is monotonic in the $p_T$ of the initial unquenched jet.
%
In that case, the $n^{\rm th}$ highest energy jet in a heavy-ion sample is a modified version of the $n^{\rm th}$ highest energy jet in the corresponding proton-proton sample.
%
Thus, in this simplified picture of energy loss, we can obtain a sample of heavy-ion jets that is statistically equivalent to its proton-proton counterpart by selecting jets with the same (upper) cumulative effective cross-section:
%
\begin{equation}
	\label{eq:cumXSdef}
\Sigma^{\rm eff}(p^{\rm min}_T) = \int_{p^{\rm min}_T}^{\infty} \df p_T \, \frac{\df \sigma^{\rm eff}}{\df p_T}.
\end{equation}
%
Note that for comparison to proton-proton cross-sections, heavy-ion cross-sections must be rescaled by the average number of nucleon-nucleon collisions $\langle N_{\rm coll} \rangle$: $\sigma^{\rm eff}_{\rm pp} = \sigma_{\rm pp}$, $\sigma^{\rm eff}_{\rm AA} = \sigma_{\rm AA}/\langle N_{\rm coll} \rangle$.
%
Of course, energy loss is not strictly monotonic in $p_T$, since other properties of a jet and of the jet-medium interaction influence its energy loss and cause jets with the same initial $p_T$ to lose different fractions of their energy.
%
Below, we will quantify the usefulness of this quantile picture in the context of a realistic event generator where significant non-monotonicities are indeed present.

Due to the steeply-falling jet production spectrum ($\sigma \sim p_T^{-6}$), jets within a given range in reconstructed heavy-ion $p_T$ are dominated by those which were least modified (see e.g.~\cite{Andrews:2018jcm}).
%
Addressing this issue requires comparing jets that had the same $p_T$ when they were initially produced.
%
In rarer events where an energetic $\gamma$ or $\Z$ boson is produced back-to-back with a jet, the unmodified boson energy approximates the initial energy of the recoiling jet \cite{Chatrchyan:2012gt,Sirunyan:2017jic}.
%
In general jet events, however, the jet energy before medium effects cannot be measured.

A key result of this work is that the quantile picture also provides a natural proxy for the unmodified jet $p_T$ that is observable in general jet events.
%
Given a heavy-ion jet with reconstructed momentum $\pTHI$, we can define $\pTq$ implicitly as the momentum of a proton-proton jet with the same (upper) cumulative cross-section:
%
\begin{equation}
	\label{eq:pTquant}
	\Sigma^{\rm eff}_{\rm pp}(\pTq) \equiv \Sigma^{\rm eff}_{\rm AA}(\pTHI).
\end{equation}
%
In this quantile picture, $\pTq$ is viewed as the initial jet $p_T$ prior to medium effects.
%
The mapping from $\pTHI$ to $\pTq$ is illustrated by the red arrows in \Fig{fig:quant_b}, with $\pTHI = \pTq \, Q_{\rm AA}(\pTq)$.
%
Intriguingly, we will show that $\pTq$ approximates the $p_T$ of a heavy-ion jet before quenching with comparable fidelity to the unmodified boson energy $\pTZ$ available only in rarer $\Z$+jet events. 
%
In particular, comparing properties of proton-proton and heavy-ion jet samples with the same $\pTq$ may substantially enhance the sensitivity of modification observables by targeting jets that were more strongly modified.


\begin{figure*}
\subfloat[\label{fig:pTloss_a}]{%
  \includegraphics[width=.5\linewidth]{figure2a.pdf}%
}
\subfloat[\label{fig:pTloss_b}]{%
  \includegraphics[width=.5\linewidth]{figure2b.pdf}%
}         
\caption{Distributions of (a) $R_{\rm AA}$ as a function of $p_T^{\rm jet}$ and (b) $Q_{\rm AA}$ as a function of $\pTq$, for the $\Z$+jet (dashed) and di-jet (solid) samples in \textsc{Jewel}.
%
Although $R_{\rm AA}$ and $Q_{\rm AA}$ are derived from the same underlying jet $p_T$ spectra, they provide different and complementary information.
%
For example, the $p_T$ dependence of $R_{\rm AA}$ is very different for $\Z$+jet and di-jet events in \textsc{Jewel}, while the average fractional $p_T$ loss $1-Q_{\rm AA}$ is similar.
%
Note that $R_{\rm AA}$ requires binning of the data, while $Q_{\rm AA}$, which is based on the cumulative cross-section, can be plotted unbinned.
}
\label{fig:pTloss}
\end{figure*}

For the remainder of this work, we consider samples of $\Z$+jet and di-jet events in the heavy-ion Monte Carlo event generator \textsc{Jewel} 2.1.0 \cite{Zapp:2013vla,KunnawalkamElayavalli:2016ttl}, based on vacuum jet production in \textsc{Pythia} 6~\cite{Sjostrand:2006za}.
%
For each process, we generate $2$ million each of proton-proton and head-on ($0-10\%$ centrality) heavy-ion events at $\SI{2.76}{TeV}$ and reconstruct anti-$k_t$ jets using \textsc{FastJet} 3.3.0 \cite{Cacciari:2008gp,Cacciari:2011ma} with radius parameter $R=0.4$ and pseudorapidity $|\eta|<2$.
%
We include initial state radiation but do not include medium recoils, since medium response is not expected to have a significant effect on \Eq{eq:cumXSdef} at the values of $p_T^{\rm min}$ considered here.
%
For $\Z$+jet events we identify the $\Z$ from its decay to muons and consider the leading recoiling jet, and for di-jet events we consider the two highest-$p_T$ jets.
%
We consider $\Z$+jet instead of $\gamma$+jet events to avoid introducing additional cuts to isolate prompt photons which could bias the validation.
%
The default heavy-ion background in \textsc{Jewel} is a Bjorken expanding medium with initial peak temperature $T_i = \SI{485}{MeV}$ and formation time $\tau_i=\SI{0.6}{fm}$, consistent with the parameters used to fit data at $\SI{2.76}{TeV}$ in more realistic hydrodynamic simulations \cite{KunnawalkamElayavalli:2016ttl,Shen:2012vn}.

Using these $\Z$+jet and di-jet samples from \textsc{Jewel}, \Fig{fig:pTloss_a} shows the standard $R_{\rm AA}$ (also called $I_{\rm AA}$ for $\Z$+jet) and \Fig{fig:pTloss_b} shows the $p_T$ ratio $Q_{\rm AA}$.
%
Although the $R_{\rm AA}$ for $\Z$+jet and di-jet events have significantly different $p_T$-dependence, it is interesting that the average fractional energy loss of jets is very similar, as quantified by $1 - Q_{\rm AA}$.
%
This might be surprising since $\Z$+jet and di-jet events have different fractions of quark and gluon jets, though \Ref{Apolinario:2018rhj} suggests that quark and gluon jets may experience similar energy loss in \textsc{Jewel}; whether this is borne out in data is an open question.
%
Regardless, it is clear that $R_{\rm AA}$ and $Q_{\rm AA}$ offer complementary probes of the jet quenching phenomenon and are therefore both interesting observables in their own right. 
%
The quantile procedure also shows that the highest-$p_T$ jets lose a small fraction of their energy on average ($(1 - Q_{\rm AA}) \sim 5\%$), even though $R_{\rm AA}$ is far below one.
%
This result can be compared to other methods for extracting the average energy loss from data, for example \Ref{He:2018gks}.

\begin{figure*}
\subfloat[\label{fig:pT-vs-pTin_a}]{%
  \includegraphics[width=.5\linewidth]{figure3a.pdf}%
}
\subfloat[\label{fig:pT-vs-pTin_b}]{%
  \includegraphics[width=.5\linewidth]{figure3b.pdf}%
}
\caption{
Mean of the jet $p_T$ distribution compared to a baseline initial $p_T$ (top), along with the corresponding standard deviation (bottom).
%
Shown are (a) $\Z$+jet events where the baseline is the physically observable $p_T$ of the recoiling $\Z$ boson and (b) di-jet events where the baseline is the unphysical and unobservable $p^{\rm MC}_T$ of the initial hard scattering obtained from \textsc{Jewel}.
%
The reconstructed jet $p_T$ for proton-proton and heavy-ion jets are shown in dashed black and blue, respectively.
%
The $\pTq$ of the heavy-ion sample, shown in red, more closely matches the initial jet $p_T$ than the reconstructed heavy-ion $p_T$ does.}
\label{fig:pT-vs-pTin}
\end{figure*}

We now turn to validating the interpretation of $\pTq$ as a proxy for the initial $p_T$ of a heavy-ion jet before quenching by the medium.
%
In $\Z$+jet events, $\pTZ$ can be used as a baseline for the (approximate) initial $p_T$ of the leading recoiling jet, since the $\Z$ boson does not interact with the quark-gluon plasma.
%
For a given value of $\pTZ$, there is a distribution of recoil jet momenta whose mean is shown in the upper panel of \Fig{fig:pT-vs-pTin_a}.
%
Even in proton-proton collisions, the recoiling jet $p_T$ is systematically lower on average than $\pTZ$ due to out-of-cone radiation and events with multiple jets.
%
In heavy-ion collisions, it is even lower due to energy loss.
%
Intriguingly, the mean value of $\pTq$ (red) is much more comparable to that of $\pTvac$ (dashed black) than $\pTHI$ (blue) is, indicating that $\pTq$ is a good proxy for the initial jet $p_T$. 
%
On the other hand, the standard deviation of $\pTq$, shown in the lower panel of \Fig{fig:pT-vs-pTin_a}, is higher than that of $\pTvac$ due to energy loss fluctuations.
%
These cannot be undone by the quantile procedure, which can only give a perfect reconstruction of the distribution of $\pTvac$ in the case of strictly monotonic energy loss.


We emphasize that the distribution in \Fig{fig:pT-vs-pTin_a} is physically observable and could be used to validate the quantile procedure in experimental data.
%
Crucially, quantile matching can also provide a baseline for the initial jet $p_T$ in general jet events.
%
To validate this in di-jet events at the generator level, we use the $p_T$ of the partons from the initial hard matrix element in \textsc{Jewel}, $\pTMC$, as an (unphysical and unobservable) baseline for the initial jet $p_T$ (see \cite{Milhano:2015mng}).
%
We consider the two highest-$p_T$ jets and match each jet with the $\pTMC$ that minimizes $\Delta R = \sqrt{\Delta \eta^2 + \Delta \phi^2}$ between the jet and the parton.
%
Each of the two jets then enters independently in \Fig{fig:pT-vs-pTin_b}, which demonstrates the correlation of the jet $p_T$ to $\pTMC$ for proton-proton and heavy-ion jets, with the results of the quantile procedure in red.
%
\Fig{fig:pT-vs-pTin_b} is the only figure in this work that involves an unobservable quantity, and it shows remarkably similar features to \Fig{fig:pT-vs-pTin_a} which can be measured experimentally.


It might be surprising that the curves in \Fig{fig:pT-vs-pTin} are fairly flat as a function of the baseline initial $p_T$.
%
This can be understood, however, from a minimal model in which the final energy of a jet is obtained from its initial energy via gaussian smearing.
%
Consider the probability distribution
%
\begin{multline}
	\label{eq:pTHI-given-pTin}
	p( \pTHI | \pTin) = \int \df \pTvac \,  \mathcal{N}( \pTHI | \tilde{\mu}_2 \pTvac,\tilde{\sigma}_2 \pTvac) \\
	\times \mathcal{N}(\pTvac | \tilde{\mu}_1 \pTin,\tilde{\sigma}_1 \pTin).
\end{multline}
%
Here, $\mathcal{N}(x | \mu,\sigma)$ is a normal distribution in the variable $x$ with mean $\mu$ and standard deviation $\sigma$, and $\tilde{\mu}_{1,2}$ and $\tilde{\sigma}_{1,2}$ are dimensionless constants.
%
\Eq{eq:pTHI-given-pTin} describes the probabilistic relation between the seed-parton momentum $\pTin$ (interpreted as $\pTZ$ or $\pTMC$) and the quenched momentum $\pTHI$ via two stages of gaussian smearing: first from $\pTin$ to the unquenched jet momentum $\pTvac$, and then from $\pTvac$ to the quenched momentum $\pTHI$.
%
Integrating over intermediate values of $\pTvac$ gives $p(\pTHI|\pTin)$, the probability of $\pTHI$ for fixed $\pTin$.
%
This is an example of a model in which the average energy loss is monotonic in $p_T$, since $\mu_2 = \tilde{\mu}_2 \, \pTvac$ is a monotonic function of $\pTvac$, but energy loss is not monotonic in $p_T$ jet-by-jet since $\tilde{\sigma}_2 \neq 0$.

The mean and standard deviation of the distribution in \Eq{eq:pTHI-given-pTin} can be calculated analytically (see \cite{Bishop:2006:PRM:1162264}): 
%
\begin{align}
\begin{split}
	\label{eq:mean-and-std}
	\langle \pTHI/ \pTin \rangle &= \tilde{\mu}_1 \, \tilde{\mu}_2,\\
	\sigma \left( \pTHI/ \pTin \right) &= \sqrt{ \tilde{\mu}_1^2 \, \tilde{\sigma}_2^2 + \tilde{\mu}_2^2 \, \tilde{\sigma}_1^2 + \tilde{\sigma}_1^2 \, \tilde{\sigma}_2^2},
	\end{split}
\end{align}
%
though the resulting distribution is not generally gaussian.
%
These can be compared to the upper and lower panels, respectively, of \Fig{fig:pT-vs-pTin}.
%
The fact that \Eq{eq:mean-and-std} has no $\pTin$-dependence is consistent with the fact that the curves in \Fig{fig:pT-vs-pTin} are approximately flat.
%
To the extent that this model is semi-realistic, \Eq{eq:mean-and-std} and a measurement of \Fig{fig:pT-vs-pTin_a} would provide an estimate of the average energy loss and the size of energy loss fluctuations.
%
Taking approximate values from \Fig{fig:pT-vs-pTin_a} at $\pTZ = \SI{300}{GeV}$ of $\langle \pTvac/ \pTZ \rangle \equiv \tilde{\mu}_1 \approx 0.87$, $\sigma \left( \pTvac/\pTin \right) \equiv \tilde{\sigma}_1 \approx 0.2\,\tilde{\mu}_1 = 0.17$, $\langle \pTHI/ \pTZ \rangle \approx 0.74$, and $\sigma \left( \pTHI/ \pTZ \right) \approx 0.24 \, \langle \pTHI/ \pTZ \rangle = 0.18$, \Eq{eq:mean-and-std} yields $\tilde{\mu}_2 \approx 0.85$ and $\tilde{\sigma}_2 \approx 0.12$.
%
It is satisfying that this extracted $\tilde{\mu}_2$ value is comparable to $Q_{\rm AA}$ in \Fig{fig:pTloss_b}, which is a more direct proxy for fractional energy loss.


\begin{figure}[t]
\includegraphics[width=7.5cm]{figure4.pdf}
\captionsetup{font=small,justification=raggedright}
\caption{\label{fig:moverpT} Distribution of $m/p_T$ for proton-proton (dashed black) and heavy-ion (blue) jets in di-jet events with reconstructed $p_T \in [100,200]$~GeV. Heavy-ion jets with $\pTq \in [100,200]$~GeV, corresponding to $\pTHI \in [80,173]$~GeV, are in red.
%
The heavy-ion result is normalized to match the proton-proton baseline but the quantile result has the correct normalization by construction.
%
Partially compensating for $p_T$ migration via the quantile procedure shifts $m/p_T$ towards being less modified.}
\end{figure}

As a final application in this Letter, we demonstrate how the quantile procedure can be used to characterize the effects of $p_T$ migration via an example jet substructure observable, the dimensionless ratio of the jet mass to its reconstructed $p_T$, $m/p_T$.
%
\Fig{fig:moverpT} shows distributions of $m/p_T$ for proton-proton and heavy-ion jets in a range of reconstructed $p_T$ in dashed black and blue, respectively.
%
Heavy-ion jets with that range of $\pTq$ are those in the same quantile as the proton-proton baseline, and $m/p_T$ for that sample is shown in red.
%
For the purpose of this example, we define $m/p_T$ from the reconstructed jet mass and $p_T$, such that the effect of the quantile procedure is only to change the $p_T$ range of jets in the selection.
%
Using the quantile procedure to (partially) account for the migration of jets to lower $p_T$, the red distribution shifts toward $m/p_T$ being less modified.
% 
We note that the jet mass is known to have significant corrections from medium response \cite{KunnawalkamElayavalli:2017hxo,Park:2018acg} so this should be taken only as an illustrative example.


In conclusion, we introduced a new strategy for comparing heavy-ion jets to a baseline of proton-proton jets in the same quantile when sorted by $p_T$. 
%
As shown in \Fig{fig:pTloss}, our new $Q_{\rm AA}$ observable is based on the same jet $p_T$ spectra as $R_{\rm AA}$ but exposes different and complementary information. 
%
As shown in \Fig{fig:pT-vs-pTin}, our new $\pTq$ observable is closely correlated with the initial $p_T$ a heavy-ion jet had before energy loss to the plasma.
%
Thus, the quantile procedure provides a data-driven way to study the modification of quenched jets and minimize the effects of sample migration.
%
Experimental tests in $\Z$+jet or $\gamma$+jet can validate the effectiveness of $\pTq$ as a proxy for the initial $p_T$ of a heavy-ion jet.
%
If these tests are successful, the quantile procedure can then be used to re-analyze measurements of jet modification observables in general jet events with an aim toward characterizing and minimizing $p_T$ migration effects and thus compare jet samples that were born alike. 
%
The measurement of $Q_{\rm AA}$ will provide information on the functional form of the average energy loss which would further constrain theoretical models.
%
It can also be used to measure differences in average energy loss between quark- and gluon-dominated jet samples.
%
Measurements of $Q_{\rm AA}$ with jet grooming \cite{Butterworth:2008iy,Ellis:2009me,Krohn:2009th,Dasgupta:2013ihk,Larkoski:2014wba} may also elucidate, for example, how energy is lost by the hard core of a jet compared to the diffuse periphery.
%
It would also be interesting to study the application of this procedure to understanding energy loss fluctuations.
%
Finally, \Fig{fig:quant} shows two additional observables---the pseudo-ratio $\widetilde{R}_{\rm AA}$ and pseudo-quantile $\widetilde{Q}_{\rm AA}$---which may be relevant for experimental applications.


\begin{acknowledgments}
We thank Liliana Apolin\'ario, Yang-Ting Chien, Yen-Jie Lee, William Lewis, James Mulligan, Aditya Parikh, Krishna Rajagopal, Gavin Salam, Andrew Turner, and Korinna Zapp for helpful discussions.
%
JB and JT are supported by the U.S. Department of Energy, Office of Science, Office of Nuclear Physics under grant Contract Number DE-SC0011090 and the Office of High Energy Physics under grant Contract Number DE-SC0012567. 
%
JT is also supported by the Simons Foundation through a Simons Fellowship in Theoretical Physics, and he thanks the Harvard Center for the Fundamental Laws of Nature for hospitality while this work was completed.
GM is supported by Funda\c c\~ao para a Ci\^encia e a Tecnologia (Portugal) under project CERN/FIS-PAR/0022/2017, and he gratefully acknowledges the hospitality of the CERN theory group.
\end{acknowledgments}

\bibliographystyle{apsrev} 
\bibliography{quantile}

\end{document}
