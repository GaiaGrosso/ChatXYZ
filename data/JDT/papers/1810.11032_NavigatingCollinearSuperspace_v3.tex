\documentclass[12pt,document,nofootinbib,superscriptaddress,onecolumn,preprintnumbers,balancelastpage]{article}
\pdfoutput=1

\usepackage{jheppub_1}

\usepackage{subfig}
\usepackage[countmax]{subfloat}
\usepackage[framemethod=TikZ]{mdframed}
\usepackage{mathtools}
\usepackage{multirow}

\usepackage{slashed}

\usepackage{booktabs} 

\setcounter{tocdepth}{2} 

\renewcommand{\textfraction}{0.10}
\renewcommand{\topfraction}{0.90}
\renewcommand{\bottomfraction}{0.90}
\renewcommand{\floatpagefraction}{0.65}

\usepackage{array}
 
\makeatletter  
\newcommand{\thickhline}{%
    \noalign {\ifnum 0=`}\fi \hrule height 1pt
    \futurelet \reserved@a \@xhline
}

\newcolumntype{"}{@{\hskip\tabcolsep\vrule width 1pt\hskip\tabcolsep}}

\makeatother

\usepackage{latexsym}
\usepackage{amssymb}
\usepackage{epsfig,amsmath,graphics}
\usepackage{listings}
\usepackage{color}
\usepackage{dcolumn}
\usepackage{slashed}
\usepackage{cancel}
\usepackage{comment,latexsym} 
\usepackage{bm}% bold math
\usepackage{verbatim}
\usepackage{tabularx}
\usepackage{dcolumn}% Align table columns on decimal point
\definecolor{Mahogany}{rgb}{0.62,0.24,0.15}
\definecolor{colorLink}{rgb}{0.7,0,0}
\definecolor{colorCite}{rgb}{0,.7,0}
\definecolor{colorURL}{rgb}{0,0,0.7}
\usepackage[colorlinks=true,linktocpage=true,linkcolor=black,citecolor=colorCite,urlcolor=colorURL]{hyperref}
\usepackage{setspace}
\usepackage{xspace}
\usepackage{multirow}
\usepackage{titlesec}
\usepackage[bbgreekl]{mathbbol}
\usepackage{bm}
%\usepackage{indentfirst}
\usepackage{float}
\usepackage{placeins}
\usepackage{afterpage}
\usepackage{footmisc}
\usepackage{breakcites}
%\usepackage{empheq}
\usepackage[most]{tcolorbox}
\tcbset{highlight math style={enhanced,colframe=green!30!black,colback=white,arc=5pt,boxrule=1.5pt}}

%%%%%%%  Added these in v48!!!   %%%%%%%%%%%%%%%%%%%%%
\usepackage[toc,page]{appendix}

\usepackage{etoolbox}
\appto\appendix{\addtocontents{toc}{\protect\setcounter{tocdepth}{1}}}

\usepackage{scalerel}
\DeclareMathOperator*{\Bigcdot}{\scalerel*{\cdot}{\bigodot}}
%%%%%%%%%%%%%%%%%%%%%%%%%%%%%%%%%%%%%%%%

\usepackage{arydshln}

\allowdisplaybreaks


% change spacing above and below equations
\expandafter\def\expandafter\normalsize\expandafter{%
    \normalsize
    \setlength\abovedisplayskip{8pt}
    \setlength\belowdisplayskip{8pt}
    \setlength\abovedisplayshortskip{8pt}
    \setlength\belowdisplayshortskip{8pt}
}


\newcommand{\OO}{\mathcal{O}}
\newcommand{\LL}{\mathcal{L}}
\newcommand{\HH}{\mathcal{H}}
\newcommand{\NN}{\mathcal{N}}
\newcommand{\DD}{\mathcal{D}}
\newcommand{\ZZ}{\mathbb{Z}_2}

\newcommand{\st}{$^\text{st}$}
\newcommand{\nd}{$^\text{nd}$}
\newcommand{\rd}{$^\text{rd}$}

\newcommand{\Li}{{\rm Li}}
\newcommand{\wt}{\widetilde}
\renewcommand{\Re}{{\rm \,Re\!}}
\renewcommand{\Im}{{\rm \,Im}}
\newcommand{\abs}[1]{\left|\, #1 \,\right|}

\newcommand{\rpii}{{\kappa_\text{I}}}
\newcommand{\rpiiii}{{\kappa_\text{III}}}



\newcommand{\ecf}[2]{e_{#1}^{(#2)}} 
\newcommand{\ecfnobeta}[1]{e_{#1}} 

\newcommand{\eref}[1]{Eq.~(\ref{#1})} 
\newcommand{\sref}[1]{Sec.~\ref{#1}} 
\newcommand{\tref}[1]{Table.~\ref{#1}} 
\newcommand{\fref}[1]{Fig.~\ref{#1}} 

\newcommand{\Dobs}[2]{D_{#1}^{(#2)}} 
\newcommand{\Dobsnobeta}[1]{D_{#1}} 

\definecolor{darkblue}{rgb}{0,0,0.5}
\newcommand{\ajl}[1]{\textbf{\textcolor{darkblue}{(#1 --ajl)}}}
\newcommand{\fm}[1]{\textbf{\textcolor{red}{(#1 --fm)}}}
\newcommand{\ms}[1]{\textbf{\textcolor{green}{(#1 --ms)}}}

\definecolor{colorTC}{rgb}{.2,.7,.2}
\newcommand{\TC}[1]{{\bf \color{colorTC}{[TC: #1]}}}
\newcommand{\GE}[1]{{\bf \color{red}{[GE: #1]}}}
\newcommand{\AL}[1]{{\bf \color{blue}{[AL: #1]}}}
\newcommand{\jdt}[1]{{\bf \color{orange}{[JDT: #1]}}}

% small space
\newcommand{\s}{\hspace{0.8pt}}

\newcommand{\nbar}{{\bar n}}
\newcommand{\scetii}{SCET$_{\rm II}$}
\newcommand{\sceti}{SCET$_{\rm I}$}
\newcommand{\nllp}{NLL$'$}
\newcommand{\qq}{\mathcal{Q}} 

\newcommand{\PP}{\mathbb{d}}

\newcommand{\oPpb}{ \bar{\mathbb P}_{\!\! n \! \perp} }
\newcommand{\oPp}{ {\mathbb P}_{\!\! n \! \perp} }

%%
%Added macros for transformation arrows
\newcommand{\RPIi}{\,\,\xrightarrow[\hspace{5pt}\text{RPI-I}\hspace{5pt}]{}\,\,}
\newcommand{\RPIii}{\,\,\xrightarrow[\hspace{5pt}\text{RPI-II}\hspace{5pt}]{}\,\,}
\newcommand{\RPIiii}{\,\,\xrightarrow[\hspace{5pt}\text{RPI-III}\hspace{5pt}]{}\,\,}
%%
\newcommand{\gauge}{\,\,\xrightarrow[\hspace{5pt}\text{Gauge}\hspace{5pt}]{}\,\,}
\newcommand{\susy}{\,\,\xrightarrow[\hspace{5pt}\text{SUSY}\hspace{5pt}]{}\,\,}
%%

\def\nslash{n\hspace{-2mm}\slash}
\def\nbarslash{\bar n\hspace{-2mm}\slash}
\def\nslashinline{n\!\!\!\slash}
\def\nbarslashinline{\bar n\!\!\!\slash}
\def\nbar{\bar n}

\DeclareRobustCommand{\Sec}[1]{Sec.~\ref{#1}}
\DeclareRobustCommand{\Secs}[2]{Secs.~\ref{#1} and \ref{#2}}
\DeclareRobustCommand{\App}[1]{App.~\ref{#1}}
\DeclareRobustCommand{\Tab}[1]{Table~\ref{#1}}
\DeclareRobustCommand{\Tabs}[2]{Tables~\ref{#1} and \ref{#2}}
\DeclareRobustCommand{\Fig}[1]{Fig.~\ref{#1}}
\DeclareRobustCommand{\Figs}[2]{Figs.~\ref{#1} and \ref{#2}}
\DeclareRobustCommand{\Eq}[1]{Eq.~(\ref{#1})}
\DeclareRobustCommand{\Eqs}[2]{Eqs.~(\ref{#1}) and (\ref{#2})}
\DeclareRobustCommand{\Ref}[1]{Ref.~\cite{#1}}
\DeclareRobustCommand{\Refs}[1]{Refs.~\cite{#1}}

\newcommand{\Nsub}[2]{\tau_{#1}^{(#2)}}
\newcommand{\Nsubnobeta}[1]{\tau_{#1}}


\newcommand{\tr}{\text{tr}}
\newcommand{\jet}{\text{jet}}
\newcommand{\event}{\text{event}}

\newcommand{\ang}[1]{\tau^{(#1)}}
\newcommand{\bang}[1]{\tau_{b}^{(#1)}}

\newcommand{\Lang}[1]{s^{(#1)}}

\newcommand{\C}[2]{C^{(#2)}_{#1}}
\newcommand{\Cnobeta}[1]{C_{#1}}


\DeclareRobustCommand{\order}[1]{{\cal O}(#1)}

\newcommand{\D}{\mathbb{D}}
\newcommand{\Dbar}{\bar{\mathbb{D}}}
\newcommand{\Q}{\mathbb{Q}}
\newcommand{\RCA}{\boldsymbol{\Omega}}


\newcommand{\be}{\begin{equation}}
\newcommand{\ee}{\end{equation}}
\newcommand{\bea}{\begin{eqnarray}}
\newcommand{\eea}{\end{eqnarray}}

\newcommand{\nbsb}{\bar{n}\cdot \bar{\sigma}} 
\newcommand{\alc}{\mathcal{A}} 
\newcommand{\alcs}{\mathcal{A}^*} 
\newcommand{\nsb}{n\cdot \bar{\sigma}} 
\newcommand{\nbs}{\bar{n}\cdot \sigma} 
\newcommand{\ns}{n\cdot \sigma} 
\newcommand{\spp}{\sigma \cdot \partial_{\perp}} 
\newcommand{\sdp}{\sigma \cdot \Delta_{\perp}} 
\newcommand{\nbp}{\bar{n}\cdot \partial} 
\newcommand{\np}{n\cdot \partial}
\newcommand{\dpp}{\Delta_\perp \cdot \partial} 
\newcommand{\bs}{\bar{\sigma}} 
\newcommand{\nD}{ n\cdot D} 
\newcommand{\nbD}{ \bar{n}\cdot D} 
\newcommand{\del}{\bigtriangledown} 
\newcommand{\PhiA}{\Phi_{\alc}}
\newcommand{\bPhiA}{\bold \Phi_{\alc}}


\newcommand{\bPhi}{\boldsymbol{\Phi}}
\newcommand{\bPsi}{\boldsymbol{\Psi}}

\newcommand{\bV}{\boldsymbol{V}}
\newcommand{\uu}{\tilde u}


\newcommand{\bM}{ \boldsymbol M}
\newcommand{\bD}{ \boldsymbol V_{\bold D}}
\newcommand{\bPhialc}{ \bold \Phi_\alc}



\DeclareRobustCommand{\order}[1]{{\cal O}(#1)}

\preprint{MIT--CTP 5076}

\title{
Navigating Collinear Superspace
}


\author[a]{Timothy Cohen,}
\author[b]{Gilly Elor,}
\author[c]{Andrew J.~Larkoski,}
\author[d,e]{and Jesse Thaler\,}

\affiliation[a]{\footnotesize Institute of Theoretical Science, University of Oregon, Eugene, OR 97403, U.S.A.}
\affiliation[b]{\footnotesize Department of Physics, Box 1560, University of Washington, Seattle, WA 98195, U.S.A.}
\affiliation[c]{\footnotesize Physics Department, Reed College, Portland, OR 97202, U.S.A.}
\affiliation[d]{\footnotesize Center for Theoretical Physics, Massachusetts Institute of Technology, Cambridge, MA 02139, U.S.A.}
\affiliation[e]{\footnotesize Department of Physics, Harvard University, 17 Oxford Street, Cambridge, MA 02138, U.S.A.}

\emailAdd{tcohen@uoregon.edu}
\emailAdd{gelor@uw.edu}
\emailAdd{larkoski@reed.edu}
\emailAdd{jthaler@mit.edu}

\abstract{
%
We introduce a new set of effective field theory rules for constructing Lagrangians with $\mathcal{N} = 1$ supersymmetry in collinear superspace.
%
In the standard superspace treatment, superfields are functions of the coordinates $\big(x^\mu,\theta^\alpha, \theta^{\dagger \dot{\alpha}}\big)$, and supersymmetry preservation is manifest at the Lagrangian level in part due to the inclusion of auxiliary $F$- and $D$-term components.
%
By contrast, collinear superspace depends on a smaller set of coordinates $\big(x^\mu,\eta,\eta^\dagger\big)$, where $\eta$ is a complex Grassmann number without a spinor index.
%
This provides a formulation of supersymmetric theories  that depends exclusively on propagating degrees of freedom, at the expense of obscuring Lorentz invariance and introducing inverse momentum scales.
%
After establishing the general framework, we construct collinear superspace Lagrangians for free chiral matter and non-Abelian gauge fields.
%
For the latter construction, an important ingredient is a superfield representation that is simultaneously chiral, anti-chiral, and real; this novel object encodes residual gauge transformations on the light cone.
%
Additionally, we discuss a fundamental obstruction to constructing interacting theories with chiral matter; overcoming these issues is the subject of our companion paper, where we introduce a larger set of superfields to realize the full range of interactions compatible with $\mathcal{N} = 1$.
%
Along the way, we provide a novel framing of reparametrization invariance using a spinor decomposition, which provides insight into this important light-cone symmetry.
%
}

\begin{document} 
\maketitle

\setcounter{page}{2}
\begin{spacing}{1.2}

\pagebreak

%%%%%%%%%%%%%%%%%%
\section{Casting Off}
\label{sec:intro}
%%%%%%%%%%%%%%%%%%
Supersymmetry (SUSY) is a powerful tool for exploring formal aspects of field theory, including Seiberg duality~\cite{Seiberg:1994pq}, Seiberg-Witten~\cite{Seiberg:1994aj,Seiberg:1994rs}, AdS/CFT for $\mathcal{N}=4$ SUSY~\cite{Maldacena:1997re}, supersymmetric localization~\cite{Pestun:2007rz, Erickson:2000af}, on-shell recursion applied to SUSY theories~\cite{Bianchi:2008pu}, and more \cite{Komargodski:2009rz,Festuccia:2011ws,Kahn:2015mla,Ferrara:2015tyn,Dall'Agata:2016yof,Delacretaz:2016nhw,Cacciatori:2017qyd}.
%
Therefore, new formulations of SUSY are of great interest in their own right, especially when they can expose new formal features.
%
Theories with $\mathcal{N}=1$ SUSY can be expressed in superspace~\cite{Salam:1974yz,Ferrara:1974ac}, which makes SUSY manifest at the Lagrangian level by relying on non-propagating field content (including the auxiliary $F$- and $D$-terms).
%
A conventional formalism for systematically extending the superspace formalism to theories with $\mathcal{N} > 1$ SUSY is not known, in part because of complications associated with a proliferation of auxiliary fields, although progress has been made in harmonic and projective superspaces~\cite{Galperin:2001uw, Galperin:1985ec,Akulov:1988tm,Howe:1998jw,Davgadorj:2017ezp,Galperin:1984av,Sokatchev:1985tc,Ivanov:1984ut,Ohta:1985au,Ivanov:2003nk}.


There do exist superspace formulations that involve only propagating physical degrees of freedom, specifically in $\mathcal{N}=4$ SUSY~\cite{Mandelstam:1982cb, Brink:1982pd, Brink:1982wv}.
%
These constructions, however, are typically discovered by starting with a component Lagrangian and then guessing a superspace formulation that reproduces the component-level result.  
%
Ideally, one would want a set of effective field theory (EFT) rules for how to put together the strictly propagating degrees of freedom into superspace Lagrangians such that SUSY would be made manifest.
%
In this paper, we realize this goal for $\mathcal{N} = 1$ SUSY theories that do not require non-propagating $F$- and $D$-term auxiliary fields to model their interactions: free chiral matter and (non-)Abelian gauge theories.
%
A companion paper will provide the necessary formalism to realize theories with non-zero $F$- and $D$-terms, such as Wess-Zumino models and gauge theories with chiral matter~\cite{Cohen:2019gsc}.


Progress towards an on-shell EFT for SUSY was recently made in \Refs{Cohen:2016jzp, Cohen:2016dcl}, where the interplay of SUSY with Soft-Collinear Effective Theory (SCET)~\cite{Bauer:2000yr, Bauer:2001ct, Bauer:2001yt} was studied.
%
By introducing the formulation of ``collinear superspace,'' these papers arrived at a ``SUSY SCET" Lagrangian with only light-cone degrees of freedom.\footnote{Collinear superspace is closely related to light-cone superspace, which has a long history.
%
It was famously utilized to prove the UV finiteness of $\mathcal{N} = 4$ SYM \cite{Mandelstam:1982cb, Brink:1982pd, Brink:1982wv}.
%
Additional work has illustrated the utility of formulating various SUSY theories \cite{Belitsky:2004yg} (and even supergravity \cite{Kallosh:2009db}) on the light cone and more \cite{Green:1996um,Maldacena:1997re,Belitsky:2004yg,Kallosh:2009db,Hearin:2010dw,Ramond:2009hb}.
%
Notably, much of the original light-cone superspace literature was focused on representation theory and implications for extended objects (see, e.g., \cite{Siegel:1981ec,Brink:1981nb}), whereas the focus of our present work on collinear superspace is identifying the bottom-up rules to construct EFTs.
}
%
The logic used in \Refs{Cohen:2016jzp, Cohen:2016dcl} was decidedly \emph{top down}: start with a full-theory Lagrangian, integrate out non-propagating degrees of freedom directly in superspace, and truncate to leading power.
%
While this was a useful first step (since it is non-trivial to show that SUSY and SCET can be compatible), any self-respecting effective field theorist would only be satisfied by a fully \emph{bottom up} treatment: specify the building blocks, define their power countings and transformation properties under the relevant symmetries, and construct the (sub)leading-order Lagrangian directly, all without appealing to an underlying full theory.


In this work, we present a concrete set of rules to construct $\mathcal{N}=1$ SUSY Lagrangians directly in collinear superspace.
%
The key insight is to make only a subgroup of $\mathcal{N}=1$ SUSY manifest and to replace full Lorentz symmetry with reparametrization invariance (RPI) \cite{Manohar:2002fd, Becher:2014oda}.
%
Note that without any loss of information, the ordinary superspace coordinate $\theta^\alpha$ can be expressed as
%
\begin{align}
\label{eq:theta_as_eta_tildeeta}
\theta^\alpha = \xi^\alpha\s \eta + {\tilde{\xi}}^{\alpha} \s\tilde{\eta}\,,
\end{align}
%
where $\xi^\alpha$ and $\tilde{\xi}^\alpha$ are orthogonal commuting basis spinors that satisfy $\xi^\alpha \s\tilde{\xi}_\alpha = 1$, and $\eta$ and $\tilde{\eta}$ are complex Grassmann numbers.
%
Then, to reduce to collinear superspace, we simply set
%
\begin{align}
\label{eq:tilde_eta_zero}
\tilde{\eta} = 0 \quad \Longrightarrow \quad \theta^\alpha = \xi^\alpha\s \eta\,,
\end{align}
%
which halves the number of manifest supercharges.
%
By exploiting the RPI freedom to rotate $\xi^\alpha$ and $\eta$, we will show that this construction preserves enough Lorentz invariance to maintain the full $\mathcal{N}=1$ SUSY at the $S$-matrix level.\footnote{The restriction in \Eq{eq:tilde_eta_zero} is reminiscent of on-shell superspace \cite{Gates:1982an, Brink:1980cb, Elvang:2013cua}, with the important distinction that our construction does not require the component fields to be exactly on-shell, \emph{i.e.}, $p^2 = m^2$ is not enforced.}


With the replacement in \Eq{eq:tilde_eta_zero}, the superspace coordinate now has the unfamiliar property that $\theta^\alpha \theta_{\alpha} = \theta^\dagger_{\dot{\alpha}} \theta^{\dagger \dot{\alpha}} = 0$.
%
This means that one cannot include $F$- and $D$-term components in a superfield, at least not in the standard way, nor can one include non-propagating components of a  spin-1/2 matter field.
%
Therefore, if a self-consistent theory of collinear superspace exists with standard superfields, it must only involve propagating degrees of freedom.
%
We will show that this is indeed the case, and the choice in \Eq{eq:tilde_eta_zero} corresponds to expressing the theory with respect to a light-like direction $n^\mu = \tilde{\xi}\s \sigma^\mu\s \tilde{\xi}^\dagger$.
%
The choice of $n^\mu$ corresponds to an explicit breaking of Lorentz invariance, leading to a set of low-energy RPI constraints.  For example, the following rescaling
\begin{align}
\xi^\alpha \,\,\rightarrow\,\, e^{- \kappa/2}\, \xi^\alpha\,, \qquad \eta \,\,\rightarrow\,\, e^{\,\kappa/2}\, \eta\,,
\end{align}
%
is known as RPI-III, which acts like an (imaginary) internal $R$-symmetry that leaves $\theta^\alpha$ unchanged.
%
By imposing collinear SUSY, RPI, and simple power counting based on mass dimension, we can construct the unique gauge-invariant EFT of free chiral superfields and (non-)Abelian vector superfields at leading power.

Another set of RPI transformations, known as RPI-II, acts to rotate $\tilde{\eta}$ into $\eta$,  which is clearly incompatible with the projection in \Eq{eq:tilde_eta_zero}.
%
In order to have a fully Lorentz-invariant theory, however, RPI-II must also be preserved.
%
Because RPI-II transforms out of the collinear SUSY algebra, we can only test RPI-II on component fields, not directly on superfields.%
%
\footnote{An analogous situation arises when writing $\mathcal{N}=2$ Lagrangians in $\mathcal{N}=1$ superspace.  While this makes the $\mathcal{N}=1$ SUSY subgroup manifest, the full $\mathcal{N}=2$ algebra can only be tested on components.}
%
For the constructions in this paper, we find that RPI-II is an accidental symmetry that is only respected by  interactions that are leading-order in mass dimension. 
%
By the Haag-\L opusza\'nski-Sohnius extension~\cite{Haag:1974qh} of the Coleman-Mandula theorem~\cite{Coleman:1967ad}, this implies that these leading-ordering constructions exhibit the full $\mathcal{N}=1$ SUSY for the $S$-matrix, even though only collinear SUSY is manifest at the Lagrangian level.
%
Establishing RPI-II for higher-order terms is more subtle, though, with complications arising when one tries to use only standard superfields. 
%
In the companion paper~\cite{Cohen:2019gsc}, we introduce novel superfields which have non-trivial RPI transformation rules and which incorporate $F$- and $D$-term components, allowing for a description of the full range of $\mathcal{N} = 1$ interactions.


Though some of the discussion here is just a bottom-up recapitulation of the top-down physics already in \Refs{Cohen:2016jzp, Cohen:2016dcl} (with an emphasis on RPI in collinear superspace), there is a crucial new ingredient.
%
Gauge theories in collinear superspace are most naturally expressed in light-cone gauge with $\bar{n} \cdot A = 0$.
%
Without a full gauge symmetry, there seemed to be no easy way to constrain the EFT interactions to ensure gauge invariance without appealing to the full-theory Lagrangian.
%
As we will describe below, the light-cone gauge condition leaves a residual gauge redundancy.
%
Remarkably, this is encoded in a new type of superfield that is simultaneously chiral, anti-chiral, and real:
%
\vspace{-3mm}
\begin{align}
\D\s \RCA = 0, \qquad \bar{\D}\s \RCA = 0, \qquad \RCA = \RCA^\dagger,
\end{align}
%
where $\D$ and $\bar{\D}$ are covariant derivatives (without spinor indices) in collinear superspace, see \Eq{eq:defDDBar} below.
%
In ordinary superspace, such a field would just be a constant; in collinear superspace, this field is only constant along the light cone.
%
A residual gauge transformation encoded by $\RCA$ is sufficient to enforce gauge invariance for both the Abelian and non-Abelian cases.
%
It is intriguing to speculate that a similar object could help illuminate the structure of light-cone supergravity.


The main result of this work is to show that -- given transformation rules governed by RPI-I, RPI-III, collinear SUSY, and residual gauge redundancy -- it is possible to construct an interesting subset of collinear SUSY theories, namely those whose interactions do not require non-propagating auxiliary degrees of freedom.
%
We will show that RPI-II is obscured by choosing a fixed light cone to define collinear superspace, though we then go on to verify that RPI-II does not yield any useful constraints on the theories studied here, at least for the leading-order interactions.
%
Foreshadowing, RPI-II will impose non-trivial constraints in our companion paper~\cite{Cohen:2019gsc}, which deals with interacting theories that require the reintroduction of the non-propagating degrees of freedom.



The rest of this paper is organized as follows.
%
In \Sec{sec:formalism}, we introduce our formalism for constructing an on-shell superspace organized around \Eq{eq:tilde_eta_zero}, and we discuss the SUSY charges, transformations, and multiplets that manifest in such a constrained setup.
%
Next, we show how these ingredients transform under RPI in \Sec{sec:RPI}.
%
In \Sec{sec:BuldingL}, we show that the kinetic term for a chiral multiplet in collinear superspace is unique.
%
\Sec{sec:Gaugetheory} then applies analogous logic to Abelian and non-Abelian gauge theories.
%
Finally, \Sec{sec:Outlook} provides an outlook.
%
A more technical discussion of RPI is provided in an appendix, followed by an appendix summarizing some useful formulae. 


%%%%%%%%%%%%%%%%%%
\section{Charting Collinear Superspace}
\label{sec:formalism}
%%%%%%%%%%%%%%%%%%
Our goal is to define a reduced $\mathcal{N} = 1$ collinear superspace which eliminates non-propagating degrees of freedom from the Lagrangian.
%
This construction will make heavy use of light-cone projections with spinors, allowing us to consistently remove half of superspace.
%
We can then construct collinear superfields that only involve the complex Grassmann coordinate $\eta$.\footnote{In all that follows, we use the mostly minus metric, the two-component spinors conventions of~\Ref{Dreiner:2008tw}, and  SUSY conventions defined in pages 449--453 of~\Ref{Binetruy:2006ad}.}

%%%%%%%%%%%%%%%%%%
\subsection{The Light Cone in Spinor-Helicity Formalism}
\label{subsec:spinor_proj}

To define standard light-cone coordinates, one introduces two light-like directions $n^\mu$ and $\bar{n}^\mu$ which satisfy $n \cdot \bar{n} = 2$.
%
We then perform a spinor-helicity decomposition in terms of two bosonic spinors $\xi^\alpha$ and $\tilde{\xi}^\alpha$\,: 
%
\begin{equation}
\label{eq:twospinorsn_alt}
n_{\alpha \dot{\alpha}} \equiv \left( \frac{n\cdot \sigma}{2} \right)_{\alpha \dot{\alpha}} = \tilde{\xi}_\alpha\, \tilde{\xi}^{\dagger}_{\dot{\alpha}}\,, \qquad \bar{n}_{\alpha \dot{\alpha}} \equiv \left( \frac{\bar{n}\cdot \sigma}{2} \right)_{\alpha \dot{\alpha}} = \xi_\alpha\, \xi^{\dagger}_{\dot{\alpha}}\,,
\end{equation}
%
or, equivalently:%
%
\footnote{Throughout this work, we suppress spinor indices when the structure is obvious, and when no confusion with scalars can arise.}
%
\begin{align}
\label{eq:twospinorsn}
n^\mu = \tilde{\xi}^\dagger\s \bar{\sigma}^\mu\s \tilde{\xi} = \tilde{\xi} \s\sigma^\mu\s \tilde{\xi}^\dagger, \quad\qquad \bar{n}^\mu = \xi^\dagger\s \bar{\sigma}^\mu\s \xi  = \xi\s \sigma^\mu\s \xi^\dagger\,.
\end{align}
%
Because the spinors are bosonic, they satisfy
\begin{equation}
\xi^\alpha\s \xi_\alpha = \tilde{\xi}^\alpha \tilde{\xi}_\alpha = 0\,.
\end{equation}
%
We choose the normalization condition
%
\begin{equation}
\xi^\alpha\s \tilde{\xi}_\alpha = 1\,,
\end{equation}
%
which ensures the desired normalization for $n^\mu$ and $\bar{n}^\mu$ via a Fierz identity:
%
\begin{align}
\label{eq:chiXiCondition}
n \cdot \bar{n} = n^\mu \,\bar{n}_\mu = 2 \, n^{\alpha \dot{\alpha}}\, \bar{n}_{\alpha \dot{\alpha}} = 2\,\Big(\xi \s\tilde{\xi} \Big)\Big( \tilde{\xi}^\dag\s \xi^\dag \Big) = 2\,.
\end{align}
%
Note that $ \tilde{\xi}^ \alpha\s \xi_\alpha = - \epsilon^{ \beta \alpha}\, \xi_\alpha\s  \tilde{\xi}_\beta  = -1$.
The standard RPI transformations correspond to all possible shifts in $n^\mu$ and $\bar{n}^\mu$ such that \Eq{eq:chiXiCondition} is maintained. We derive a version of RPI that constrains possible operators in collinear superspace in \Sec{sec:RPI}.

When it is convenient to choose an explicit reference frame, a common choice is to align $n^\mu$ and $\bar{n}^\mu$ along the $z$-direction.
%
This \emph{canonical frame} is specified by 
%
\begin{align}
n^\mu = (1,0,0,1)\,, \qquad \bar{n}^\mu = (1,0,0,-1)\,,
\end{align}
which is equivalent to the fixing the spinors to\footnote{Note that this is consistent due to the unfortunate fact that $\epsilon^{12} = - \epsilon^{21} = \epsilon_{21} = - \epsilon_{12} = 1$.}
\begin{align}
\label{eq:bosoniccanonicalspinors}
\xi^\alpha = (0,1)\,,\qquad \xi_\alpha = (-1,0)^{\intercal}, \qquad \tilde{\xi}^\alpha = (1,0)\,, \qquad \tilde{\xi}_\alpha = (0,1)^{\intercal}\,,
\end{align}
as can be verified using \Eq{eq:twospinorsn}.
%
As we show below, this frame choice is equivalent to working in the collinear superspace frame developed in~\Ref{Cohen:2016jzp}.

Any operator can be projected along the $\xi_\alpha$ and $\tilde{\xi}_\alpha$ spinor axes.
%
Consider the differential operator $\sigma^\mu\, \partial_\mu$, where $\partial_\mu$ is the four-vector partial derivative.
%
We can construct differential operators on the light-cone as
%
\begin{align}
\label{eq:Funnyd}
\PP &=  \bar{n} \cdot \partial =  \xi^\alpha (\sigma \cdot \partial)_{\alpha \dot{\alpha}} \, \xi^{\dagger \dot{\alpha}} \, ,  &
\tilde{\PP} &= n \cdot \partial  = \tilde{\xi}^\alpha (\sigma \cdot \partial)_{\alpha \dot{\alpha}} \, \tilde{\xi}^{\dagger \dot{\alpha}} \, ,\notag\\[8pt]  
\PP_\perp & =  \xi^\alpha (\sigma \cdot \partial)_{\alpha \dot{\alpha}} \, \tilde{\xi}^{\dagger \dot{\alpha}} \, , &
\PP_\perp^* & = \tilde{\xi}^\alpha (\sigma \cdot \partial)_{\alpha \dot{\alpha}} \, \xi^{\dagger \dot{\alpha}}. 
\end{align}
%
Here, we have introduced the $\PP$ notation to emphasize that we have \emph{not} made a specific frame choice.\footnote{Note that $\PP / \tilde{\PP}$, which are equivalent to $\nbp/ \np$, are often referred to in the literature as $\partial_{\pm}$.  See \Ref{Leibbrandt:1983pj} for a review on standard light-cone conventions. We adopt the new $\PP/\tilde{\PP}$ convention to emphasize that we can formulate the theory without appealing to a specific frame. We are unaware of any light-cone-independent analog of $\PP_\perp$ in the literature.}
%
The d'Alembertian can be expressed along an unspecified light-cone direction as
%
\begin{align}
\Box = \PP \s \tilde{\PP} - \PP_\perp^* \PP_\perp \, .
\end{align}


%%%%%%%%%%%%%%%%%%
\subsection{Projecting Spinors and Gauge Fields}
\label{sec:ProjectGauge}

Throughout this paper, we make use of the light-cone decomposition of fields that carry Lorentz indices.
%
We begin by discussing the light-cone projections for a left-handed two-component Weyl spinor $u_\alpha$.
%
Recall that $u_\alpha$ may be decomposed (by acting with chiral projection operators) onto a helicity component that is aligned with the light cone and another that is anti-aligned.%
%
\footnote{This decomposition is valid for both massless and massive fermions, though in the massive case, the helicity components are not mass eigenstates.}
%
Specifically, we can decompose
\vspace{-3mm}
\begin{align}
\label{eq:udecom}
&u_\alpha =  \tilde{\xi}_\alpha\s u - \xi_\alpha\s \tilde{u}\,, 
\end{align}
with
\vspace{-3mm}
\begin{align}
\label{eq:udecom_reverse}
 \xi^\alpha \s u_\alpha = u \quad \text{and} \quad \tilde{\xi}^\alpha\s  u_\alpha = \tilde{u}\,.
\end{align}
%
Here, $u$ is the helicity component that propagates in collinear superspace, while $\uu$ is the other helicity which will play a role in~\Ref{Cohen:2019gsc}.
%

Next, we decompose the full Lorentz four-vector field $A^\mu$ as
%
\begin{align}
\left(\sigma \cdot A \right)_{\alpha \dot{\alpha}} = \xi_\alpha\, \xi^{\dagger}_{\dot{\alpha}}\, n\cdot A + \tilde{\xi}_\alpha\, \tilde{\xi}^\dagger_{\dot{\alpha}}\, \bar{n}\cdot A+\sqrt{2} \,  \xi_\alpha\, \tilde{\xi}^\dagger_{\dot{\alpha}}\, \alc^* +\sqrt{2} \,  \tilde{\xi}_\alpha\, \xi^{\dagger}_{\dot{\alpha}}\, \alc\,,
\end{align}
%
where we have projected the gauge field $A^\mu$ onto a complex ``light-cone gauge" scalar using
%
\begin{align}
\label{eq:def_script_A}
\alc =\frac{1}{\sqrt{2}} \xi^\alpha\s (\sigma \cdot A )_{\alpha \dot{\alpha}}\s \tilde{\xi}^{\dagger \dot{\alpha}} \, , \quad \quad \alc^* =  \frac{1}{\sqrt{2}} \tilde \xi^\alpha\s (\sigma \cdot A )_{\alpha \dot{\alpha}}\s \xi^{\dagger \dot{\alpha}} \,.
\end{align}
This $\alc$ field encodes the two propagating degrees of freedom of a gauge field, \emph{i.e.}, those that are transverse to the light cone. 

The two other degrees of freedom, $n\cdot A$ and $\bar{n} \cdot A$, while non-propagating (and therefore not the focus of the current work) can be obtained via the projections 
\begin{align}
\bar n \cdot A = \xi^\alpha\s (\sigma \cdot A)_{\alpha \dot{\alpha}}\s \xi^{\dot{\alpha} \dagger} \, , \quad \quad  n\cdot A = \tilde \xi^\alpha\s (\sigma \cdot A)_{\alpha \dot{\alpha}}\s \tilde \xi^{\dagger \dot{\alpha}} \,.
\label{eq:nAandnBarA}
\end{align}
%
The $\bar{n}\cdot A$ mode may be eliminated by enforcing light-cone gauge, as will be done in what follows.
%
Furthermore, it is straightforward to see that no light-cone time derivatives act on $n\cdot A$, and as such it can be treated as a non-propagating component of the gauge field.
%
It is therefore prudent to integrate it out using the equations of motion, which yields the well known light-cone Lagrangian for the gauge field, see \emph{e.g.}~\Ref{Leibbrandt:1983pj}.  


The Lagrangians constructed in \Secs{sec:BuldingL}{sec:Gaugetheory} will involve only the propagating degrees of freedom:  $\phi$, $u$, and $\alc$.
%
As we will see in \Sec{subsec:RPIcomponents}, however, RPI-II transforms us away from our chosen slice of collinear superspace.
%
For this reason, it will often be convenient to make RPI-II manifest by introducing auxiliary degrees of freedom:  $\uu$, $n \cdot A$, and $\bar{n}\cdot A$.
%
We have just shown that these fields correspond to projections of the full Lorentz representations $u^\alpha$ and $A^\mu$, so we can derive their RPI properties from the ``top down'' using the Lorentz algebra (see \App{app:RPIgen}).
%
That said, we will construct the actual Lagrangians from the ``bottom up,'' relying on the auxiliary fields only to check for possible RPI-II constraints on the low-energy effective theory.
%
At the end of the day, we will find that RPI-II does not introduce any additional requirements on the theories studied here.


%%%%%%%%%%%%%%%%%%
\subsection{Projecting Superspace Coordinates}

We can now use the light-cone spinors to isolate half of superspace.
%
Starting from the standard $\mathcal{N} = 1$ superspace coordinate $\theta^\alpha$, we can construct two spinor projections:%
%
\footnote{For later convenience, we have chosen a different sign convention for the projection of a superspace coordinate than for the projection of a spinor field in \Eq{eq:udecom}.}
%
\begin{align}
\label{eq:eta_def}
\begin{array}{l}\eta =  \tilde{\xi}^\alpha\, \theta_\alpha \\[5pt] \tilde{\eta} = -\xi^\alpha\, \theta_\alpha \end{array}\,, \qquad \Longleftrightarrow \qquad \theta^\alpha = \xi^\alpha\, \eta + \tilde{\xi}^{\alpha}\, \tilde{\eta}\,,
\end{align}
where $\eta$ and $\tilde{\eta}$ are complex Grassmann numbers which \emph{do not} carry a spinor index.
%
Note that the minus sign in \Eq{eq:eta_def} results from the identity $ \xi \s \tilde{\xi}= 1 = - \tilde{\xi} \s\xi $.
%
The conjugate superspace coordinates are defined analogously:  
%
\vspace{-1mm}
\begin{align}
\begin{array}{l}\eta^\dagger =  - \tilde{\xi}^\dagger_{\dot{\alpha}}\, \theta^{\dagger\dot{\alpha}}\\[5pt]  \tilde{\eta}^\dagger = \xi^\dagger_{\dot{\alpha}}\, \theta^{\dagger\dot{\alpha}}\end{array}\,, \qquad \Longleftrightarrow \qquad \theta^{\dagger\dot{\alpha}} = \eta^\dagger \, \xi^{\dagger \dot{\alpha}} + \tilde{\eta}^\dagger \, \tilde{\xi}^{\dagger \dot{\alpha}}\,.
\end{align}
%
It is helpful to note that $\big(\xi^\alpha\s \tilde{\xi}_\alpha\big)^\dagger = \big(\tilde{\xi}_\alpha\big)^\dagger \big(\xi^\alpha\big)^\dagger = \tilde{\xi}^\dagger_{\dot{\alpha}}\s \xi^{\dagger \dot{\alpha}} =1 = - \xi^\dagger_{\dot{\alpha}}\s \tilde{\xi}^{\dagger \dot{\alpha}}$. 
As expected from their anti-commuting nature, one can verify that $\eta^2 = \big(\eta^\dagger\big)^2 = \big\{\eta, \eta^\dagger \big\} = 0$.
%
Crucially, $\xi_\alpha$ and $ \xi^{\dagger}_{\dot{\alpha}}$ are complex conjugates of each other, such that the superfield $\boldsymbol{\Phi}^\dagger$ will be the conjugate of $\boldsymbol{\Phi}$ (see \Eqs{eq:chiralsuperfield}{eq:antichiralsuperfield} below).
%
We choose as convention for the mass dimension 
\begin{align}
\big[\,\xi\,\big] = 0\,, \qquad \big[\,\eta\,\big] = -1/2 \,,
\end{align}
such that the standard mass dimension $\big[\,\theta\,\big] = -1/2$ is maintained.

We can perform a similar decomposition of the supercoordinate derivative: 
%
\begin{align}
\frac{\partial \eta\,}{\,\partial \theta^\alpha} = \tilde{\xi}^\alpha\quad \text{and} \quad \frac{\partial \tilde{\eta}\,}{\,\partial \theta^\alpha} =  - \xi^\alpha \qquad \Longrightarrow \qquad \frac{\partial}{\partial \theta_\alpha} = \tilde{\xi}^\alpha \frac{\partial}{\partial \eta} - \xi^\alpha \frac{\partial}{\partial \tilde{\eta}}\,\,.
\end{align}
%
This is consistent with the anti-commutation relations:
\begin{equation}
\left\{\eta,\, \frac{\partial}{\partial \eta}  \right\} = 1\,, \qquad \left\{\tilde{\eta},\, \frac{\partial}{\partial \tilde{\eta}}  \right\} = 1\,, \qquad \left\{\eta, \,\frac{\partial}{\partial \tilde{\eta}}  \right\} = 0\,, \qquad \left\{\tilde{\eta}, \,\frac{\partial}{\partial \eta}  \right\} = 0\, . 
\end{equation}
%
Now that $\eta$ and $\tilde{\eta}$ are factorized, reducing to collinear superspace is as simple as
\begin{align}
\Aboxed{\begin{minipage}{0.09\linewidth}\vspace{7pt}$\hspace{7pt}\tilde{\eta} = 0$\vspace{7pt}\end{minipage}} \qquad \Longrightarrow  \qquad \theta^\alpha = \xi^\alpha\, \eta\,, \qquad \frac{\partial}{\partial \theta_\alpha} =  \tilde{\xi}^\alpha \frac{\partial}{\partial \eta}\,.
\end{align}
With this restriction, it follows that $\theta^\alpha \theta_{\alpha} = 0$, implying that the usual $F$- and $D$-term auxiliary fields must be absent in this setup (see \Sec{sec:Superfields}). 

%%%%%%%%%%%%%%%%%%
\subsection{The Collinear SUSY Algebra}

Using this light-cone spinor decomposition, the supercharges and superspace derivatives take a simple form.
%
Starting from the full $\mathcal{N}=1$ SUSY algebra,%
%
\footnote{We use the superscript ``full" to be explicit when working with objects of the full $\mathcal{N}=1$ theory, or in situations where we have not yet restricted to collinear superspace, \emph{i.e.} set $\tilde{\eta} = 0$.} 
\begin{align}
 \Big \{ \qq_\alpha^{\rm full}, \qq^{\dagger \rm full}_{\dot{\alpha}} \Big \} =  -2\s i\, (\sigma \cdot \partial)_{\alpha \dot{\alpha}}  \, ,
\end{align}
we can construct various sub-algebras by contracting with the $\xi^\alpha$ and $\tilde{\xi}^\alpha$ spinors.
%
For instance, contracting with $\xi^\alpha$ and $\xi^\dagger_{\dot{\alpha}}$, we obtain
\begin{align}
\label{eq:reducedSUSYalg}
 \Big \{ \Q^{\rm full}, \Q^{\dagger \rm full} \Big \} =  - 2\s i\, \PP\,,  \qquad \text{with} \qquad \Q^{\rm full} \equiv \xi^\alpha\, \qq^{\rm full}_\alpha\,, \qquad \Q^{\dagger \rm full} \equiv \qq^{\dagger \rm full}_{\dot{\alpha}}\, \xi^{\dagger \dot{\alpha}}\,,
\end{align}
%
where $\Q^{\rm full}$ and $\Q^{\dagger \rm full}$ are collinearly-projected SUSY generators without spinor indices.
%
The collinear sub-algebra in \Eq{eq:reducedSUSYalg} will be the focus of this study.

Without loss of generality, the original SUSY generators can be expressed in terms of the $\eta$ and $\tilde{\eta}$ coordinates as  
\begin{align}
\qq_\alpha^\text{full} =i \s \tilde{\xi}_\alpha \,\frac{\partial}{\partial \eta} - i\s \xi_\alpha\, \frac{\partial}{\partial \tilde{\eta}} -  (\sigma \cdot \partial)_{\alpha \dot{\alpha}} \big(\eta^\dagger\, \xi^{\dagger \dot{\alpha}} + \tilde{\eta}^\dagger\, \tilde{\xi}^{\dagger \dot{\alpha}}\big) \, ,
\end{align}
and similarly for $\big(\qq^{ \text{full}}_{\alpha}\big)^\dagger$.
%
Using the definition of $\Q$ in \Eq{eq:reducedSUSYalg}, this yields
%
\begin{align}
\Q^\text{full} = i\s \frac{\partial}{\partial \eta}  - \eta^\dagger\, \PP - \tilde{\eta}^\dagger\, \PP_\perp\,. 
\end{align}
%
Note that the $\PP_\perp$ term does not contribute to the anti-commutator in \Eq{eq:reducedSUSYalg} since $\big(\Q^{ \text{full}}\big)^\dagger$ depends on $\partial/\partial \eta^\dagger$, not on $\partial/\partial \tilde{\eta}^\dagger$.

To restrict to collinear superspace, we simply set $\tilde{\eta} = 0$.
%
The collinear SUSY generators are now
%
\begin{align}
\Q \equiv \xi^\alpha \qq_\alpha^{\rm full} \bigg|_{\tilde{\eta} = 0}   = i\s \frac{\partial}{\partial \eta} -  \eta^\dagger \,\PP, \qquad \Q^\dagger \equiv \qq^{\dagger \rm full}_{\dot{\alpha}}   \,\xi^{\dagger \dot{\alpha}} \bigg|_{\tilde{\eta} = 0} = i\s \frac{\partial}{\partial \eta^\dagger} - \eta\, \PP\,.
\label{eq:defCollinearSUSYGens}
\end{align}
%
Even with this restriction, the collinear versions of $\Q$ and $\Q^\dagger$ still satisfy \Eq{eq:reducedSUSYalg}, \emph{i.e.}, $\big\{ \Q, \Q^{\dagger} \big \} =  - 2\s i\, \PP\,$. 
%
Note that $\PP$ commutes with both collinear SUSY generators: 
%
\begin{align}
\Big[ \PP, \Q \Big] = 0 = \Big[ \PP, \bar{\Q} \Big]\,.
\end{align}
%
When using the canonical frame in \Eq{eq:bosoniccanonicalspinors}, this sub-algebra is equivalent to the collinear superspace algebra in \Refs{Cohen:2016jzp, Cohen:2016dcl}, given by
\begin{align}
\Big \{\Q, \Q^\dagger \Big \}  =  \Big\{ \qq_2, \qq^\dagger_{\dot{2}} \Big\} \,, \quad \quad  \,\,\, \Big\{\qq_2, \qq^\dagger_{\dot{1}} \Big\} &=\Big\{\qq_1, \qq^\dagger_{\dot{2}} \Big\} =\Big\{\qq_1, \qq^\dagger_{\dot{1}} \Big\}  = 0\,.
\end{align}
%
Closure of this sub-algebra will be discussed \Sec{sec:SUSYTrans}.

For completeness, we note that other projections yield
\begin{align}
& \Big \{\xi^\alpha\, \qq_\alpha, \qq^\dagger_{\dot{\alpha}}\, \tilde{\xi}^{\dagger \dot{\alpha}} \Big \}  = - 2\s i\, \PP_\perp \, , \quad  \Big \{\tilde{\xi}^\alpha\, \qq_\alpha, \qq^\dagger_{\dot{\alpha}}\, \xi^{\dagger \dot{\alpha}} \Big \}  = - 2\s i\, \PP_\perp^* \, , \\[8pt] \nonumber
& \hspace{60pt}  \Big \{\tilde{\xi}^\alpha\, \qq_\alpha, \qq^\dagger_{\dot{\alpha}}\, \tilde{\xi}^{\dagger \dot{\alpha}} \Big \}  = - 2\s i\, \tilde{\PP} \, ,  
\end{align}
corresponding to different sub-algebras of the full $\mathcal{N} = 1$ SUSY.
%
In this way, the spinors $\xi^\alpha$ and $\tilde{\xi}^\alpha$ allow us to define SUSY sub-algebras that point along the collinear, anti-collinear, and transverse directions.

%%%%%%%%%%%%%%%%%%
\subsection{Collinear Super-Covariant Derivatives}

In order to manipulate and restrict superfields, it is useful to define collinear super-covariant derivatives.
%
These can be obtained by projecting the ordinary super-covariant derivatives using the light-cone spinors.
%
Starting from the full superspace derivative
%
\begin{align}
&\mathcal{D}_\alpha^\text{full} = \frac{\partial }{\partial \theta_\alpha} - i \s (\sigma \cdot \partial)_{\alpha \dot{\alpha}}\s \theta^{\dagger \dot{\alpha}} =  \tilde{\xi}_\alpha \frac{\partial}{\partial \eta} - \xi_\alpha \frac{\partial}{\partial \tilde{\eta}} -i\s (\sigma \cdot \partial)_{\alpha \dot{\alpha}} \big(\eta^\dagger \,\xi^{\dagger \dot{\alpha}} + \tilde{\eta}^\dagger\, \tilde{\xi}^{\dagger \dot{\alpha}}\big) \, , 
\end{align}
we can reduce to collinear superspace operators by setting $\tilde{\eta} = 0$:
\begin{align}
\D \equiv \xi^\alpha \mathcal{D}_\alpha^\text{full}  \bigg|_{\tilde{\eta} = 0}  = \frac{\partial}{\partial \eta} - i\s \eta^\dagger\, \PP,  \qquad \bar{\D} \equiv \bar{\mathcal{D}}_{\dot{\alpha}}^\text{full} \xi^{\dagger \dot{\alpha}}  \bigg|_{\tilde{\eta} = 0}  =   \frac{\partial}{\partial \eta^\dagger} - i\s \eta\, \PP\, ,
\label{eq:defDDBar}
\end{align}
where these operators carry mass dimension $\big[\,\D\,\big] =  \big[\,\bar{\D}\,\big] = 1/2$. 
%
We see that
\begin{align}
\Big\{ \D, \bar{\D} \Big \}  = -2\s i\, \PP, \qquad \Big\{\D, \Q\Big\} = 0 = \Big\{\D, \Q^\dagger\Big\} = \Big\{\bar{\D}, \Q\Big\} = \Big\{\bar{\D}, \Q^\dagger\Big\},
\end{align}
so these objects behave as superspace derivatives in our constrained superspace.
%
In particular, $\D$ or $\bar{\D}$ acting on a collinear superfield yields another collinear superfield.


A number of properties of $\mathcal{D}_\alpha^\text{full}$ and $\bar{\mathcal{D}}_{\dot{\alpha}}^\text{full}$ carry over to $\D$ and $\bar{\D}$.
%
For example, one can perform integration by parts under the collinear superspace integral $\int \text{d} \eta \, \text{d}\eta^\dagger$.
%
One key difference, however, is that
%
\vspace{-2mm}
\begin{equation}
\D^2 = \bar{\D}^2 = 0\,,
\end{equation}
since we only have a single Grassmann coordinate $\eta$ after setting $\tilde{\eta} = 0$.
%
As usual, $\D$ and $\bar{\D}$ allow us to define a notion of chirality for a superfield, as will be discussed next. 

\subsection{Collinear Superfields}
\label{sec:Superfields}
A generic collinear superfield is any function of $\big(x^\mu,\eta,\eta^\dagger\big)$.
%
Here, we focus on superfields that do not carry any Lorentz indices, with the idea being that such indices could always be contracted with $\xi^\alpha$ and $\tilde{\xi}^\alpha$ to form a Lorentz scalar.%
%
\footnote{Superfields with non-trivial Lorentz structure will be utilized in the companion paper \cite{Cohen:2019gsc}.} 
%
Due to its Grassman nature, $\eta^2 = 0$, the most general bosonic scalar superfield is
%
\begin{align}
\label{eq:Snotation}
\boldsymbol{S}\big(x,\eta,\eta^\dagger\big) = a(x) + \eta \, b(x) + \eta^\dagger c(x) + \eta^\dagger \eta \, v(x)\,,
\end{align}
%
where $a$ and $v$ are complex scalar fields, $b$ and $c$ are complex Grassmann fields, and we follow the common practice of using bold font to delineate a superfield.
%
To make this look more familiar, we could instead take an ordinary superfield written in terms of $\theta^\alpha$, and just make the replacement $\theta^\alpha = \xi^\alpha \,\eta$, remembering that $\theta^2 = 0$.
%
This yields
\begin{align}
\label{eq:altSnotation}
\boldsymbol{S} = a + \eta \, \xi^\alpha\, b_\alpha + \eta^\dagger\, \xi^\dagger_{\dot{\alpha}} \,c^{\dot{\alpha}} + \eta^\dagger \eta \, \xi \big(\sigma^\mu v_\mu\big) \xi^\dagger\,,
\end{align}
where again $a$ is a complex scalar, $b_\alpha$ is a spinor, $c^{\dot{\alpha}}$ is an anti-spinor, and $v^\mu$ is a vector.
%
Of course, these different ways of writing $\boldsymbol{S}$ contain the exact same information, with $b \equiv \xi^\alpha \,b_\alpha$, $c  \equiv \xi^\dagger_{\dot{\alpha}} \,c^{\dot{\alpha}}$, and $v \equiv \xi \big(\sigma^\mu v_\mu\big) \xi^\dagger$.

From this generic collinear superfield, we can apply constraints in the usual way:
%
\begin{itemize}
\item{Chiral: $\bar{\D}\s \boldsymbol{\Phi} = 0$\,;}
\item{Anti-Chiral: $\D\s \boldsymbol{\Phi}^\dagger = 0$\,;}
\item{Real: $\boldsymbol{V} = \boldsymbol{V}^\dagger$\,.}
\end{itemize}
%
These are analogous to the representations in ordinary $\mathcal{N} = 1$ SUSY, with an important twist:  because $\bar{\D}^2 = 0$, acting a \emph{single} $\bar{\D}$ on any superfield gives a chiral superfield.
%
For the same reason, there is no notion of a linear superfield $\boldsymbol{L}$, since~$\bar{\D}^2\s \boldsymbol{L} = \D^2\s \boldsymbol{L} = 0$.


Focusing on the components of a chiral multiplet $\boldsymbol{\Phi}$,  
%
\begin{align}
\label{eq:chiralsuperfield}
\boldsymbol{\Phi}\big(x,\eta,\eta^\dagger\big) = \phi(x) + \sqrt{2} \, \eta \, u(x) + i \s \eta^\dagger \eta \, \PP \phi(x) \,,
\end{align}
%
it is clear that this representation is built from a complex scalar $\phi$ degree of freedom and a single helicity fermionic degree of freedom $u \equiv \xi^\alpha \,u_\alpha$, \emph{i.e.}\ an anti-commuting Lorentz scalar.
%
It is easy to check that the chirality condition is satisfied since $\bar{\D} \boldsymbol{\Phi} = i\s \eta \, \PP \phi - i\s \eta \, \PP \phi = 0$.
%
Similarly, an anti-chiral superfield can be written as
\begin{align}
\label{eq:antichiralsuperfield}
\bPhi^\dagger\big(x,\eta,\eta^\dagger\big) = \phi^*(x) + \sqrt{2} \, \eta^\dagger u^\dagger (x) - i \s\eta^\dagger \eta \, \PP \phi^*(x)  \, ,
\end{align}
where again $u^\dagger \equiv \xi^\dagger_{\dot{\alpha}} u^{\dagger \dot{\alpha}}$ is the propagating helicity of the fermion. 
%
Note that \Eq{eq:antichiralsuperfield} is indeed the complex conjugate of \Eq{eq:chiralsuperfield}.
%
These chiral superfields can be used as building blocks to generate additional superfields by acting on them with superspace derivatives:
\begin{align}
\label{eq:DPhicomponents}
\D\s \bPhi  &= \sqrt{2} \, u - 2 \s i\, \eta^\dagger \PP \phi - i\s \sqrt{2} \, \eta^\dagger \eta \, \PP u \,, \notag\\[5pt]
\bar{\D} \s\bPhi^\dagger  &=  \sqrt{2} \, u^\dagger - 2\s i\, \eta \, \PP \phi^* + i\s \sqrt{2} \, \eta^\dagger  \eta \, \PP u^\dagger \,.
\end{align}

Next, consider a real superfield field $\boldsymbol{V}$, written in the notation of \Eq{eq:altSnotation},
\begin{align}
\label{eq:vector}
\bV\big(x,\eta,\eta^\dagger\big) = a(x) + i\s \eta \, \xi^\alpha\, b_\alpha(x) - i\s \eta^\dagger\, \xi^\dagger_{\dot{\alpha}} \,b^{\dagger \dot{\alpha}}(x) + \eta\, \eta^\dagger \, \xi \big(\sigma \cdot v(x)\big) \xi^\dagger \, ,
\end{align}
where $a$ is a real scalar, $b_\alpha$ is a spinor, and $v^\mu$ is a real vector.
%
In the standard $\mathcal{N} = 1$ SUSY approach, real superfields are used to encode gauge fields and gauginos, but this is not possible in collinear superspace for a few reasons.
%
First, $v^\mu$ in \Eq{eq:vector} only contains one propagating degree of freedom, instead of the two helicities needed for a physical gauge field.
%
Second, $b_\alpha$ has the wrong mass dimension (and the wrong gauge transformation properties) to play the role of the gaugino.
%
Third, the usual approach to constructing the gauge field strength via $\boldsymbol{W}_\alpha = \big(\bar{\mathcal{D}}^\text{full}\big)^2 \mathcal{D}_\alpha^\text{full}\s \boldsymbol{V}$ does not work in collinear superspace because $\bar{\D}^2 = 0$.
%
A new approach is required, which is the subject of \Sec{sec:Gaugetheory}.

A key ingredient for understanding gauge theories is a new type of superfield which does not have a counterpart in ordinary superspace.
%
This is a representation that is simultaneously chiral, anti-chiral, and real:
\begin{align}
\bar{\D}\s \RCA = 0, \qquad \D\s \RCA^\dagger = 0, \qquad \RCA^\dagger = \RCA\,,
\end{align}
where the symbol $\RCA$ was chosen since this will encode residual gauge transformations in light-cone gauge.
%
The chirality condition implies that $\RCA$ can be written as 
\begin{align}
\RCA\big(x,\eta,\eta^\dagger\big) &= \omega(x) + i\s \eta\, \xi\, \psi_\omega(x) + i\s \eta^\dagger \eta \, \PP \omega(x)\,,
\end{align}
for the bosonic scalar field $\omega$ and the fermionic scalar field $\psi_\omega$.
%
The reality condition implies
\begin{align}
\label{eq:RCA}
\omega = \omega^*, \qquad \psi_\omega = 0, \qquad \PP\s \omega = - \PP\s \omega^* = -\PP\s \omega = 0 \qquad \Longrightarrow \qquad \RCA\big(x,\eta,\eta^\dagger\big) = \omega(x)\,.
\end{align}
In the full $\mathcal{N} = 1$ superspace, this would just be trivial constant superfield.
%
In collinear superspace, $\PP_\perp \omega \neq 0$; this will turn out to be exactly the component we need to encode the superfield gauge transformations.

\subsection{Collinear Superspace Translations} 
\label{sec:SUSYTrans}

Under an ordinary SUSY transformation, the superspace coordinates transform as
\begin{align}
\theta^\alpha &\susy \theta^\alpha + \zeta^\alpha \,, \nonumber\\
\theta^{\dagger \dot{\alpha}} &\susy \theta^{\dagger \dot{\alpha}} + \bar{\zeta}^{\dot{\alpha}} \, , \nonumber\\
x^\mu &\susy x^\mu +  i\s\zeta\sigma^\mu\theta^{\dagger}+i\s\bar{\zeta}\bar{\sigma}^\mu \theta \, ,
\end{align}
where $\zeta^\alpha$ is a constant two-component Grassmann spinor.
%
To capture the same information in collinear superspace, we simply make the replacement $\theta^\alpha = \xi^\alpha\, \eta$ and $\zeta^\alpha = \xi^\alpha\, \epsilon$, which gives a representation of the collinear SUSY algebra in \Eq{eq:reducedSUSYalg}:  
\begin{align}
\eta &\susy \eta + \epsilon \, , \nonumber \\ 
\eta^\dagger &\susy \eta^\dagger + \epsilon^\dagger \,, \nonumber \\ 
x^\mu &\susy x^\mu + i \s\bar{n}^\mu \big(\epsilon\, \eta^\dagger  + \epsilon^\dagger\, \eta\big)\,.
\end{align}
We emphasize that $\epsilon$, which parametrizes collinear SUSY transformations, does not carry a spinor index.

Acting on a chiral superfield from \Eq{eq:chiralsuperfield}, a collinear SUSY transformation yields
\begin{align}
\label{eq:susyvariation}
\delta_\epsilon\s \bPhi & = - i\s \big(\epsilon\, \Q + \epsilon^\dagger\, \Q^\dagger \big) \bPhi \nonumber \\[5pt]
 & = \sqrt{2} \, \epsilon \, u + 2 \s i \, \epsilon^\dagger\, \eta\,  \PP\s \phi + \sqrt{2}\s i \, \epsilon \, \eta^\dagger \eta \, \PP \s u\,,
\end{align}
from which we can deduce the component transformations, 
\begin{align}
\delta_\epsilon \phi &= \sqrt{2} \, \epsilon \, u\,, \nonumber \\
\delta_\epsilon u &= - i\s\sqrt{2} \, \epsilon^\dagger\, \PP \s\phi\,,
\label{eq:componentshift}
\end{align}
with similar results for the conjugate fields.
%
As for ordinary chiral multiplets, we can introduce a shifted spacetime coordinate to simplify SUSY manipulations:
%
\begin{equation}
y^\mu \equiv x^\mu + i \s\bar{n}^\mu\, \eta^\dagger \eta\,,  \qquad y^\mu \susy y^\mu + 2\s i\, \bar{n}^\mu\, \epsilon^\dagger\, \eta\,.
\end{equation}
%
From this, it is clear that
%
\begin{equation}
\boldsymbol{\Phi}\big(x,\eta,\eta^\dagger\big) =\boldsymbol{\Phi}\big(y,\eta\big) = \phi(y) + \sqrt{2} \, \eta \, u(y)\,,
\end{equation}
%
which gives a slightly simpler way to derive \Eq{eq:componentshift}.  That said, we will stick with the $x^\mu$ coordinates throughout this paper.

Note that the highest component of a collinear chiral superfield -- the fermionic $u$ component -- transforms as a total derivative.
%
Because it is fermionic, though, we cannot construct a collinear-SUSY-invariant action using a standard bosonic chiral superpotential.
%
In the companion paper \cite{Cohen:2019gsc}, we show how to construct a novel fermionic chiral superpotential, using fermionic chiral superfields whose highest component is bosonic.%
%
As shown in \Eq{eq:DPhicomponents}, this kind of object is what one gets from $\bar{\D} \s\bPhi^\dagger$.


Starting from a real collinear superfield from \Eq{eq:vector}, we can derive the component transformation rules:  
%
\begin{align}
\delta_\epsilon\s a &= i\! \left( \epsilon \, b - \epsilon^\dagger b^\dagger  \right)\,,  \nonumber \\
\delta_\epsilon\s b &=  - i\s \epsilon \, v + i\s \epsilon\, \PP\s a\, ,\nonumber \\
\delta_\epsilon\s v &=\epsilon\, \PP\s b + \epsilon^\dagger\, \PP\s b^\dagger\, .
\label{eq:componentshift_vector}
\end{align}
%
Here $b = \xi^\alpha b_\alpha$ and the highest component $v= v_\mu\,\xi\s \sigma^\mu\s \xi^\dagger$ is bosonic, real, and transforms as a total derivative, and we will use that to construct Lagrangians in \Secs{sec:BuldingL}{sec:Gaugetheory}.
%
For the superfield in \Eq{eq:RCA}, which is simultaneously chiral, anti-chiral, and real, it transforms as
\begin{align}
\delta_\epsilon \RCA = 2\s i\, \eta^\dagger\, \eta\, \epsilon^\dagger \,\xi^\dagger\, \PP \omega = 0\,,
\end{align}
implying that $\RCA$ is a supersymmetric object, though we have not found a way to construct nontrivial Lagrangians with it.


Finally for completeness, we can test whether the collinear SUSY sub-algebra closes, by showing that the commutator of two transformations is a spacetime translation along the light-cone direction.
%
Given two SUSY transformations $\delta_{\epsilon_1}$ and $\delta_{\epsilon_2}$ acting on the components of a chiral superfield, we find
\begin{align}
\Big[ \delta_{\epsilon_1}, \delta_{\epsilon_2} \Big] \phi &= 2  \left( \epsilon_2\, \epsilon^{\dagger}_{1} -  \epsilon_1\, \epsilon^{\dagger}_{2}\right) \PP u\,, \nonumber\\
\Big[ \delta_{\epsilon_1}, \delta_{\epsilon_2} \Big] u & =  2  \left( \epsilon_2\, \epsilon^{\dagger}_{1} -  \epsilon_1\, \epsilon^{\dagger}_{2}\right) \PP \phi \,,
\end{align}
as expected from \Eq{eq:reducedSUSYalg}.

%%%%%%%%%%%%%%%%%%
\section{A Collinear Superspace Sextant: Reparametrization Invariance}
\label{sec:RPI}

The spinor projections in \Sec{subsec:spinor_proj} naively appear to be an explicit breaking of Lorentz symmetry, since they identify a preferred light-cone direction.
%
However, this breaking is artificial:  the choice of $\xi^\alpha$ and $\tilde{\xi}^\alpha$ is arbitrary since any light-cone choice would yield the same physics.
%
The redundancy of choosing a light-cone direction encodes the underlying Lorentz structure of the theory via the RPI transformations (see \App{app:RPIgen} for details).
%
For our purposes, RPI simply enforces that the physics must be unchanged by the choice of light-cone direction and therefore that every object decomposed in light-cone coordinates must have well-defined RPI transformation properties. 

%%%%%%%%%%%%%%%%%%
\subsection{RPI Transformations of the Light Cone}

To derive the action of the RPI transformations, we need to identify transformations on $\xi^\alpha$ and $\tilde{\xi}^\alpha$ which preserve 
\begin{align}
\xi^\alpha \s\tilde{\xi}_\alpha = 1,
\label{eq:RPIcondition}
\end{align}
which is equivalent to $n\cdot \bar{n} = 2$ in \Eq{eq:chiXiCondition}.
%
The most general linear transformation on $\xi^\alpha$ and $\tilde{\xi}^\alpha$ is  
\begin{align}
\xi_\alpha \,\,&\longrightarrow\,\, a \, \tilde{\xi}_\alpha  +b \, \xi_\alpha \,,\label{eq:PreRPI-II}\\[5pt]
\tilde{\xi}_\alpha \,\,&\longrightarrow\,\, c \, \tilde{\xi}_\alpha+d\, \xi_\alpha\,, \label{eq:PreRPI-I}
\end{align}
where $a$, $b$, $c$, and $d$ are complex coefficients.
%
Maintaining \Eq{eq:RPIcondition} requires 
\begin{equation}
 a\s d-b\s c = 1.
\end{equation}
This implies that the group of transformations that maintain the normalization of the spinors are complex linear transformation with unit determinant, namely SL(2,$\mathbb{C}$).\footnote{More specifically, the group is projective because overall signs play no role.}
%
The six generators correspond to Lorentz transformations on the celestial sphere~\cite{Larkoski:2014bxa}, whose properties are reviewed in more detail in \App{app:RPIgen}.

These six transformations are usually grouped into three categories: 
% 
\begin{align}
\xi & \RPIi  \xi \,, & \tilde{\xi} &\RPIi \tilde{\xi}+ \kappa_{\rm I} \,  \xi\,, \label{eq:RPI1}\\[5pt]
\quad \quad \xi & \RPIii  \xi+ \kappa_{\rm II} \, \tilde{\xi}\,,& \tilde{\xi} &\RPIii  \tilde{\xi}\,, \label{eq:RPI2}\\[5pt]
\quad \quad \xi &\RPIiii   e^{-\kappa_{\rm III}/2} \, \xi\,, &\tilde{\xi} &\RPIiii  e^{\,\kappa_{\rm III}/2} \, \tilde{\xi}\,. \label{eq:RPI3}
\end{align}
%
While $\kappa_{\rm I}$ and $\kappa_{\rm II}$ are in general complex, we typically restrict $\kappa_{\rm III}$ to be real, since a simple phase rotation of $\xi$ and $\tilde{\xi}$ does not change $n^\mu$ or $\bar{n}^\mu$, as is clear from \Eq{eq:twospinorsn}.
%
One can also understand the reality of $\kappa_{\rm III}$  by examining the algebra given in \App{app:RPIgen}, or by recognizing that imaginary $\kappa_{\rm III}$ corresponds to the SO(2) little group.
%
Thus, there are five non-trivial RPI generators, which correspond to three different ways of maintaining \Eq{eq:RPIcondition}.
%
Taking $\xi$ to be fixed while shifting $\tilde{\xi}$ in the perpendicular direction yields RPI-I.
%
Reversing the roles of $\xi$ and $\tilde{\xi}$ yields RPI-II.
%
If both spinors transform by equal and opposite scale transformations, this yields RPI-III.
%

%%%%%%%%%%%%%%%%%%
\subsection{RPI Transformations of Projected Objects}
%
To derive the action of RPI on projected objects, we simply apply the transformations for $\xi$ and $\tilde{\xi}$, while leaving the underlying Lorentz-covariant objects unchanged.
%
The relevant RPI transformations of various objects are summarized in \Tab{table:RPI-components}.

\begin{table}[t!]
\renewcommand{\arraystretch}{1.7}
\setlength{\arrayrulewidth}{.3mm}
\centering
%\small
\setlength{\tabcolsep}{0.95 em}
\begin{tabular}{ |c || c | c | c|}
\hline
 Object &  RPI-I &  RPI-II & RPI-III   \\ \hline \hline
 $\tilde{\xi}^\alpha $  & $\tilde{\xi}^\alpha + \rpii \,\xi^\alpha$   & $\tilde{\xi}^\alpha $ &  $ e^{\rpiiii/2}\, \tilde{\xi}^\alpha $ \\
$\xi^\alpha  $ &     $\xi^\alpha$    & $\xi^\alpha+ \kappa_{\rm II} \, \tilde{\xi}^\alpha$ &  $ e^{-\rpiiii/2}\, \xi^\alpha$  \\
\hline
  $\tilde{\PP} $ &  $ \tilde{\PP} +\rpii\, \PP_\perp+\kappa_{\rm I}^*\, \PP^*_\perp$  & $\tilde{\PP}$ & $ e^\rpiiii \,\tilde{\PP}$ \\
  $\PP  $ &  $\PP$   & $\PP + \kappa_{\rm II}^* \, \PP_\perp + \kappa_{\rm II} \, \PP_\perp^*$  &$e^{-\rpiiii}\,\PP$ \\
 $\PP_\perp $ &   $ \PP_\perp + \kappa_{\rm I}^*\, \PP$   &$\PP_\perp + \kappa_{\rm II} \, \tilde{\PP} $& $ \PP_\perp $ \\
  $\PP_\perp^* $ &  $\PP_\perp^* + \rpii\, \PP$  & $\PP_\perp^* + \kappa_{\rm II}^*  \, \tilde{\PP}$& $ \PP_\perp^* $ \\
\hline
$\phi  $ &    $\phi$   &  $\phi$ &  $ \phi$  \\
  $ u $ &   $u$ & $u + \kappa_{\rm II} \, \tilde{u}$ & $  e^{-\rpiiii/2}\,u $ \\
  $ \tilde{u} $ &     $ \tilde{u} + \kappa_{\rm I} \, u$ & $\tilde{u}$ & $  e^{\rpiiii/2}\, \tilde{u} $ \\
  \hline
  ${n} \cdot A $ &  $ {n} \cdot A +\sqrt{2} \left(\rpii\, \alc +\kappa_{\rm I}^*\, \alc^*\right)$  & ${n} \cdot A$ & $ e^\rpiiii \,{n} \cdot A$ \\
  $\bar{n} \cdot A   $ &  $\bar{n} \cdot A$   & $\bar{n} \cdot A + \sqrt{2} \left(\kappa_{\rm II}^* \, \alc + \kappa_{\rm II} \, \alc^* \right)$  &$e^{-\rpiiii}\,\bar{n} \cdot A $ \\
  $\alc  $ &  $\alc+\frac{\kappa_{\rm I}^*}{\sqrt{2}} \,\bar n \cdot A$     &  $\alc + \frac{\kappa_{\rm II}}{\sqrt{2}} \, n\cdot A$ &  $\alc$  \\
$\alc^*  $ &  $\alc^*+\frac{\rpii}{\sqrt{2}}\, \bar n\cdot A$  & $\alc^* + \frac{\kappa_{\rm II}^*}{\sqrt{2}} \,n\cdot A $ & $\alc^*$  \\
\hline
\end{tabular}
% 
\caption{\small{The RPI transformation properties for various spinor projections, derivatives, and component fields.  RPI-II transformations can be derived for the component fields, but not for collinear superfields. }
\label{table:RPI-components}}
\end{table}


Under RPI-I and RPI-II, the light-cone four-vectors transform as 
\begin{align}
n^\mu &\RPIi n^\mu + \Delta_\perp^\mu \,, &  \bar{n}^\mu &\RPIi \bar{n}^\mu\,,\\[5pt]
 n^\mu &\RPIii n^\mu\,, & \bar{n}^\mu &\RPIii \bar{n}^\mu  + \epsilon^\mu_\perp\,, \label{eq:RPI2_nn}
\end{align}
where we have defined
\begin{align}
\label{eq:deltaepsilon}
\Delta^\mu_\perp = \kappa_{\rm I} \, \xi\s \sigma^\mu\s \tilde{\xi}^\dagger + \kappa_{\rm I}^* \, \tilde{\xi} \s \sigma^\mu\s \xi^\dagger \quad \text{and} \quad \epsilon^\mu_\perp = \kappa_{\rm II} \,  \xi \s\sigma^\mu \s\tilde{\xi}^\dagger + \kappa_{\rm II}^* \, \tilde{\xi}\s \sigma^\mu\s \xi^\dagger. 
\end{align}
%
The four-vectors $\Delta_\perp^\mu$ and $\epsilon_\perp^\mu$ only have non-zero components in the directions perpendicular to the light-cone, so RPI-I and RPI-II correspond to rotations around the light-cone~\cite{Marcantonini:2008qn}.\footnote{To make contact with the notation used in the SCET literature, simply replace $\Delta_\perp \cdot \partial =   \kappa_{\rm I}\, \PP_\perp + \kappa_{\rm I}^* \,\PP_\perp^* $ and  $\epsilon_\perp \cdot \partial =  \kappa_{\rm II}\, \PP_\perp^* + \kappa_{\rm II}^*\, \PP_\perp $, as can be verified using \Eqs{eq:Funnyd}{eq:deltaepsilon}.  This helps confirm that the generators we identify as RPI-I, -II, and -III correspond to the usual ones in the SCET literature.}
%
Under RPI-I and RPI-II, 
\begin{align}
\PP &\RPIi \PP \,,  & \tilde{\PP} &\RPIi \tilde{\PP} + \rpii\, \PP_\perp + \rpii^*\, \PP^*_\perp\, , \\
\PP &\RPIii\PP + \kappa_{\rm II}^*\, \PP_\perp + \kappa_{\rm II} \,\PP_\perp^* \,, & \tilde{\PP} &\RPIii \tilde{\PP} \,.
\end{align}
%
The mixed spinor derivatives transform as
%
\begin{align}
\PP_\perp &\RPIi \PP_\perp + \kappa_{\rm I}^*\, \PP \,,  & \PP_\perp^* & \RPIi \PP_\perp^*  + \kappa_{\rm I}\, \PP \,, \\
 \PP_\perp &\RPIii \PP_\perp  + \kappa_{\rm II}\, \tilde{\PP} \, , & \PP_\perp^* &\RPIii \PP_\perp^* +  \kappa_{\rm II}^*\, \tilde{\PP} \,.
\end{align} 
%


We can repeat the above logic for RPI-III, yielding
%
\begin{align}
n^\mu = \tilde{\xi}\s \sigma^\mu\s \tilde{\xi}^\dagger &\RPIiii e^{\,\kappa_{\rm III}/2}\, \tilde{\xi}\s \sigma^\mu\, e^{\,\kappa_{\rm III}/2}\, \tilde{\xi}^\dagger = e^{\,\kappa_{\rm III}}\, n^\mu,\\[5pt]
\bar{n}^\mu = \xi\s \sigma^\mu\s \xi^\dagger &\RPIiii  e^{- \kappa_{\rm III}/2}\,\xi\s \sigma^\mu \, e^{- \kappa_{\rm III}/2} \xi^\dagger = e^{- \kappa_{\rm III}}\, \bar{n}^\mu,
\end{align}
%
which correspond to boosts along the light-cone direction~\cite{Marcantonini:2008qn}.
%
Therefore, $\PP =  \bar{n} \cdot \partial$ and $\tilde{\PP} =  n \cdot \partial$ defined in \Eq{eq:Funnyd} transform as
%
\begin{equation}
\PP \RPIiii e^{- \kappa_{\rm III}}\, \PP\,, \qquad\text{and}\qquad \tilde{\PP} \RPIiii e^{\,\kappa_{\rm III}}\, \PP\,,  
\end{equation}
%
since $\partial_\mu$ is an ordinary Lorentz vector that is unaffected by RPI.
%
Note that $\PP_\perp$ and $\PP_\perp^*$ are invariant under RPI-III, as they contain both $\xi^\alpha$ and $\tilde{\xi}^\alpha$.
%
Additionally, we see that $\Box = \PP\s \tilde{\PP}  - \PP_\perp^*\s \PP_\perp$ is invariant under all of RPI-I, RPI-II, and RPI-III, which is an important consistency check.

%%%%%%%%%%%%%%%%%%
\subsection{RPI Transformations of Component Fields}
\label{subsec:RPIcomponents}

The RPI transformation properties of component fields can also be derived from the transformations on $\xi$ and $\tilde{\xi}$.
%
A scalar field $\phi$ transforms trivially.
%
The fermion field $u = \xi^\alpha u_\alpha$ is invariant under RPI-I
\begin{align}
u \RPIi u\,,
\end{align}
as can be derived using \Eq{eq:RPI1}, and transforms as 
\begin{align}
u \RPIiii e^{-\kappa_{\rm III}/2}\, u\,,
\end{align} 
as is clear from \Eq{eq:RPI3}.
%
As will be discussed below in \Sec{subsec:RPIii}, $u$ transforms into $\tilde{u} \equiv  \tilde{\xi}^\alpha \s u_\alpha$ under RPI-II: 
%
\begin{equation}
\label{eq:RPI2_on_u}
u \RPIii u + \kappa_{\rm II} \,\tilde{u}\,.
\end{equation}
%
Note that the $\tilde{u}$ field does not appear explicitly in the constructions in this paper.
%
This is connected to the fact that we will be forced to check RPI-II directly on the component Lagrangian, as explained below in \Sec{subsec:RPIii}.


As discussed in \Sec{sec:ProjectGauge}, the propagating modes of a gauge field are naturally expressed as a complex scalar 
%
\begin{equation}
\label{eq:scriptAdef}
\mathcal{A} \equiv A_\mu \, \xi\s \sigma^\mu\s \tilde{\xi}^\dagger\,.
\end{equation}
%
However, since this ``scalar'' explicitly depends on $\xi$ and $\tilde{\xi}$, it has non-trivial RPI transformations.  
%
In particular, $\mathcal{A}$ is not invariant under RPI-I or RPI-II, although it is invariant under RPI-III:
\begin{align}
\mathcal{A} &\RPIi \mathcal{A} + \frac{\kappa_{\rm I}}{\sqrt{2}} \, \bar{n}\cdot A\,,\\[2pt]
\mathcal{A} &\RPIii \mathcal{A} + \frac{\kappa_{\rm II}}{\sqrt{2}} \, n\cdot A\,,\\[2pt]
\mathcal{A} &\RPIiii \mathcal{A} \,.
\end{align}
%
In light-cone gauge where $\bar{n}\cdot A = 0$, $\mathcal{A}$ is invariant, which is useful for writing a gauge theory Lagrangian that is consistent with RPI-I transformations.
%
This observation will allow us to construct theories in collinear superspace that preserve both RPI-I and RPI-III, while again RPI-II must be checked at the component level. 


%%%%%%%%%%%%%%%%%%
\subsection{Implications for Collinear Superspace}
\label{subsec:ImpCollSup}

\begin{table}[t!]
\renewcommand{\arraystretch}{1.8}
\setlength{\arrayrulewidth}{.3mm}
\centering
\setlength{\tabcolsep}{0.95 em}
\begin{tabular}{ |c || c | c|}
    \hline
 Field &  RPI-I &   RPI-III   \\ \hline \hline
     $\eta $  & $\eta$   &    $ e^{\rpiiii/2}\, \eta $ \\
     \hline
      $\Q$ & $\Q$& $e^{-\kappa_{\rm III}/2}  \Q$ \\
     \hline
      $\D$ &$\D$ & $e^{-\kappa_{\rm III}/2} \,\D$ \\
  $\bar{\D}$ & $\bar{\D}$ & $e^{-\kappa_{\rm III}/2} \,\bar{\D}$ \\
  \hline
$\bPhi$  & $\bPhi$   &  $\bPhi$    \\
\hline
\end{tabular} 
\caption{\small{RPI transformations for various collinear superspace objects after setting $\tilde{\eta} = 0$.  Note we have not provided the RPI-II transformations since they take us outside the collinear SUSY sub-algebra. }
}
\label{table:RPI-superfields}
\end{table}

Due to the connection between SUSY and Lorentz invariance, clearly the superspace coordinates $\eta$ and $\tilde{\eta}$ must transform non-trivially under RPI.
%
Using their definitions in \Eq{eq:eta_def} and remembering that a Lorentz spinor -- specifically $\theta_\alpha$ for our purposes here -- is invariant under RPI, implies
\begin{align}
\eta &\RPIi \eta - \kappa_{\rm I} \,  \tilde{\eta} \,, &  \tilde{\eta} &\RPIi \tilde{\eta} \,, \\[2pt]
\eta &\RPIii \eta\,, &  \tilde{\eta} &\RPIii \tilde{\eta}- \kappa_{\rm II} \, \eta\,, \label{eq:RPI2_eta}\\[2pt]
\eta &\RPIiii e^{\,\kappa_{\rm III}/2}\, \eta\,,&  \tilde{\eta} &\RPIiii e^{-\kappa_{\rm III}/2} \, \tilde{\eta}  \,.
\end{align}
%
Additionally, the collinear SUSY generators $\Q$ and $\Q^\dagger$ have non-trivial transformation properties under RPI, as they must since the ordinary SUSY generators transform as Lorentz spinors. 


We immediately see that setting $\tilde{\eta} = 0$ is compatible with RPI-I and RPI-III, but not with RPI-II.
%
The reason is that the shift required by \Eq{eq:RPI2_eta} generically makes $\tilde{\eta}$ non-zero.
%
Therefore, as discussed further in \Sec{subsec:RPIii} below, we cannot make RPI-II manifest in collinear superspace.

There is a reduced RPI compatible with collinear superspace, consisting of just RPI-I and RPI-III: 
%
\begin{align}
\Aboxed{\begin{minipage}{0.185\linewidth}\vspace{7pt}$\hspace{7pt}\text{taking}\quad\tilde{\eta} = 0$\vspace{7pt}\end{minipage}} \hspace{50pt}  \eta  \RPIi \eta, \qquad   \eta \RPIiii e^{\,\kappa_{\rm III}/2} \, \eta\,.
\end{align}
%
Note that RPI-III acts like an imaginary $R$-symmetry where $\eta$ has $R$-charge $+1/2$.
%
Related transformation properties are inherited by the collinear super-covariant derivatives from \Eq{eq:defDDBar}:
%
\begin{align}
\D \RPIiii e^{-\kappa_{\rm III}/2} \,\D\,, \qquad \text{and} \qquad \bar{\D} \RPIiii e^{-\kappa_{\rm III}/2}\, \bar{\D}\,.
\end{align}
%
The RPI-I and RPI-III transformation properties of various collinear superspace objects are given in \Tab{table:RPI-superfields}.


In order for RPI-I and RPI-III to be manifest at the Lagrangian level, collinear superfields have to have well-defined transformation properties.
%
Because the lowest component of a standard chiral multiplet is a Lorentz scalar, we expect $\boldsymbol{\Phi}$ from \Eq{eq:chiralsuperfield} to be invariant under both RPI-I and RPI-III.
%
This can be verified explicitly using the component transformation properties from \Tab{table:RPI-components}.
%
For example, performing an RPI-III transformation, we find
\begin{align}
\phi &\RPIiii  \phi \, , \nonumber\\[2pt]
\eta \, u &\RPIiii  e^{\,\kappa_{\rm III}/2}\, \eta\,  e^{-\kappa_{\rm III}/2}\, u = \eta \, u \, ,\nonumber\\[2pt]
\eta^\dagger \eta \, \PP \phi &\RPIiii  e^{\,\kappa_{\rm III}/2}\,\eta^\dagger\, e^{\,\kappa_{\rm III}/2} \, \eta \, e^{- \kappa_{\rm III}}\, \PP \phi = \eta^\dagger \eta \, \PP \phi\,.
\end{align}
Note that $\kappa_{\rm III}$ is real, so $\eta^\dagger\s \eta$ is not RPI-III invariant on its own.
%
A similar calculation shows that each term of $\bPhi$ is RPI-I invariant as well.
%
We emphasize that $\bPhi$ \emph{does not} have well-defined superfield RPI-II transformation properties, though its components do.

%%%%%%%%%%%%%%%%%%
\subsection{Where is RPI-II?}
\label{subsec:RPIii}
%

Ultimately, we are interested in constructing Lorentz-invariant theories, so we want to enforce the full RPI symmetry, including RPI-II.
%
Because RPI-II and collinear SUSY do not commute, though, we cannot simultaneously realize RPI-II while imposing the defining constraint of collinear superspace:  $\tilde{\eta} = 0$.
%
The reason is that RPI-II corresponds to a translation of $\tilde{\eta}$, as can be seen in \Eq{eq:RPI2_eta}.
%
Therefore, unlike for RPI-I and RPI-III, there are no EFT rules for constructing RPI-II-invariant operators directly in collinear superspace. 


Of course, what is really going on is that RPI-II and collinear SUSY are just non-commuting sub-algebras of a larger $\mathcal{N} = 1$ structure.
%
After all, collinear SUSY (two supercharges) plus full RPI implies at least $\mathcal{N} = 1$ SUSY (four supercharges), since that is the smallest graded algebra consistent with Lorentz invariance \cite{Haag:1974qh}.
%
So while RPI-II does not map collinear superfields to collinear superfields, we can still apply the RPI-II transformations from \Tab{table:RPI-components} to component fields.
%
We will later use these component transformations to show that RPI-II is respected by the Lagrangians constructed in \Secs{sec:BuldingL}{sec:Gaugetheory}.

From \Tab{table:RPI-components}, we see that $u$ transforms into $\uu$ under RPI-II.
%
The field $\uu$ has a ``top-down" interpretation as the helicity component (of a massless spinor representation of the Lorentz algebra), which is non-propagating when we construct a theory on the light cone.
%
From the ``bottom-up'' perspective, we can view $\uu$ as a constrained fermion mode in the effective theory, whose constraint equation must be the most general one allowed by the symmetries of the theory, namely RPI-I and RPI-III.
%
These two perspectives are of course related, where the bottom-up constraint should corresponds to the top-down equation of motion used to integrate out $\uu$.


Assuming $\uu$ is linear in $u$, we can derive a constraint equation of the form $\uu + \hat{\mathcal{O}}\s u = 0$, where $\hat{\mathcal{O}}$ is some differential operator.
%
To ensure RPI-III invariance, we need all terms in this constraint equation to have the same RPI-III charge, so that we may consistently set it to zero.
%
Since $u$ ($\uu$) has RPI-III charge $-1/2$ ($+1/2$), $\hat{\mathcal{O}}$ must have RPI-III charge $+1$, which means that $\hat{\mathcal{O}}$ must be proportional to $\tilde{\PP}$ or $1/\PP$.
%
Note that $\PP_\perp$ and $\PP_\perp^*$ have no RPI-III charge, so we can use them freely as long as $\hat{\mathcal{O}}$ has mass dimension zero.


Turning to RPI-I, $u$ is inert but $\uu \to \uu + \kappa_{\rm I} \s u$, so we must have $\hat{\mathcal{O}} \rightarrow \hat{\mathcal{O}} - \kappa_{\rm I}$.
%
Assuming there are no mass scales in the problem, this uniquely fixes $\hat{\mathcal{O}} = - \PP_\perp^* / \PP$, yielding the constraint $\uu = (\PP_\perp^*/\PP) u$.%
%
\footnote{More generally, $\hat{\mathcal{O}}$ could include a term proportional to $f/\PP$ if $f$ has mass dimension one.}
%
Inserting this into \Eq{eq:RPI2_on_u} yields the RPI-II transformation: 
%
%
\begin{align}
\label{eq:uNLRPIii}
u  &\RPIii u +  \kappa_{\rm II}\s \frac{\PP_\perp^*}{\PP}\s u\,.
\end{align}
%
One can verify that this transformation is consistent with the RPI algebra given in \Eq{eq:RPIComm}.


For the massless theories in \Secs{sec:BuldingL}{sec:Gaugetheory}, we can use \Eq{eq:uNLRPIii}  to verify the RPI-II invariance of our derived Lagrangians at the component level.
%
For theories that involve additional mass scales, the RPI-II transformations of $u$ cannot be uniquely defined using the above logic.
%
This is one of the reasons why in the companion paper, instead of imposing a constraint on $\uu$, we introduce a novel superfield whose lowest component is $\uu$ \cite{Cohen:2019gsc}.
%
Here, our focus is on massless theories, so we can simply use \Eq{eq:uNLRPIii} without reference to $\uu$.%
%
\footnote{While this  realization of RPI-II might seems odd, is not unexpected given that the opposite helicity field $\tilde{u}$ is absent from the theory. Indeed, the RPI-II transformations of an uncharged fermion in SCET can be derived by demanding that the full theory fermion be invariant \cite{Cohen:2016jzp}.}

%%%%%%%%%%%%%%%%%%
\section{Learning the Ropes:  Free Chiral Multiplets}
\label{sec:BuldingL}

In the spirit of EFTs, our goal is to elucidate the underlying symmetries and power-counting rules that yield valid Lagrangians in collinear superspace.
%
We are now armed with all the necessary tools to understand what superspace operators are allowed without having to rely on matching to an explicitly Lorentz-invariant construction as in \Refs{Cohen:2016jzp, Cohen:2016dcl}.  
%
Using the ingredients from \Secs{sec:formalism}{sec:RPI}, we can construct Lagrangians directly in collinear superspace by demanding RPI-I, RPI-III, collinear SUSY, and global/gauge symmetries.
%
As emphasized in \Sec{subsec:RPIii}, imposing RPI-II requires explicit manipulation of the component Lagrangian.
%
In this section, we consider the simplest case of a single free chiral multiplet, which already illuminates many aspects of the collinear superspace formalism.

%%%%%%%%%%%%%%%%%%
\subsection{Rules for Building an Action}

The most straightforward way to impose collinear SUSY on an action is to express the Lagrangian as the lowest component of a total superspace derivative:%
%
\footnote{An equivalent way to write \Eq{eq:genericL} is \begin{align}\mathcal{L} = \int \text{d} \eta \s \text{d} \eta^\dagger \, \bV_{\rm comp} + \int \text{d} \eta \, \bPhi_{\rm comp} + \int \text{d} \eta^\dagger \,\bPhi^\dagger_{\rm comp}\,,\end{align} though we found \Eq{eq:genericL} to be more convenient for practical calculations.}
%
\begin{align}
\label{eq:genericL}
\mathcal{L} =  \Big( \D \s\bar{\D}\s \bV_{\rm comp} + \D \s\bPhi_{\rm comp} + \Dbar \s\bPhi^\dagger_{\rm comp}  \Big) \Big|_{0}\,\,,
\end{align}
%
where $\bV_{\rm comp}$ is a composite real multiplet $\big(\bV_{\rm comp} = \bV^\dagger_{\rm comp}\big)$, $\bPhi_{\rm comp}$ is a composite chiral multiplet $\big(\Dbar\s \bPhi_{\rm comp} = 0\big)$, and the zero subscript indicates the restriction to $\eta = 0 = \eta^\dag$. 
%
Here, ``composite'' means that it is constructed from elementary superfields, \emph{e.g.} a product of elementary chiral multiplets is a composite chiral multiplet.
%
Using \Eqs{eq:componentshift}{eq:componentshift_vector}, it is clear that the lowest components of $\D\s \Dbar\s \bV_{\rm comp}$ and $\D \s \bPhi_{\rm comp}$ transform as total derivatives under collinear SUSY.
%
Therefore, the action $S = \int \text{d}^4 x \, \mathcal{L}$ is invariant under collinear SUSY.
%
An analogous logic is used to justify the SUSY invariance of the standard off-shell superspace formulation of $\mathcal{N} = 1$ SUSY, see \emph{e.g.}~\cite{Bertolini:2013via}.


In this paper, we work exclusively with elementary superfields that are bosonic.
%
If $\bPhi_{\rm comp}$ is bosonic, then $\D \bPhi_{\rm comp}$ is fermionic, so it is not useful for our purposes here to include such a term in the action, \Eq{eq:genericL}. 
%
Therefore, we set $\bPhi_{\rm comp} = 0$ for the remainder of this paper, yielding the generic collinear SUSY Lagrangian
%
\begin{align}
\label{eq:VrestrictedL}
\mathcal{L} =  \D \bar{\D} \bV_{\rm comp} \Big|_{0}\,\,,
\end{align}
%
which effectively means that there is no analog for the ``superpotential'' in this construction.\footnote{If $\bPsi_{\rm comp}$ is bosonic and chiral, then $\bPhi_{\rm comp} = \Dbar \bPsi^\dagger_{\rm comp}$ is fermionic and chiral.  Inserting this $\bPhi_{\rm comp}$ into \Eq{eq:genericL} is equivalent to setting $\bV_{\rm comp} = \bPsi_{\rm comp} + \bPsi_{\rm comp}^\dagger$, and not only does it not generate a new type of term, but it in fact yields a total derivative.}

The Lagrangian must satisfy the following requirements, which impose a set of constraints on the form of $\bV_{\rm comp}$:
%
\begin{itemize}
\item \textbf{Mass dimension four}:
%
Recall that $\big[\D\big] = \big[\bar{\D}\big] = 1/2$, which implies that $\bV_{\rm comp}$ must have mass dimension three.%
%
\footnote{In this paper, we perform power counting based on mass dimension, which one can show is equivalent to the SCET power counting $(\tilde{\PP}, \PP, \PP_\perp) \sim Q (\lambda^2, 1, \lambda)$ when RPI-III is taken into account.}
%
%
The kinetic term for chiral superfields is given in \Eq{eq:VcompKinetic} below, along with arguments for its validity and uniqueness.
%
\item \textbf{Lorentz invariant}:
%
Lorentz invariance of the $S$-matrix is equivalent to RPI as discussed in \Sec{sec:RPI}.
%
Since $\D$ and $\Dbar$ are invariant under RPI-I, $\bV_{\rm comp}$ must be as well.
%
The product $\D \s\Dbar$ has RPI-III charge $-1$, so $\bV_{\rm comp}$ must have RPI-III charge $+1$.
%
%
As discussed in \Sec{subsec:RPIii}, RPI-II has to be checked at the component level and cannot be directly enforced as a property of $\bV_{\rm comp}$.
%
\item \textbf{Gauge invariant}:
%
To have an RPI-I invariant action, we must fix to light-cone gauge, as discussed around \Eq{eq:scriptAdef}.
%
In \Sec{sec:Gaugetheory}, we show that a residual gauge symmetry survives in the form of a real, chiral, and anti-chiral multiplet $\RCA$.  Enforcing residual gauge symmetry yields additional constraints on the Lagrangian.
%
\end{itemize}
%
As in standard EFTs, we can write down a valid collinear SUSY action by identifying all terms consistent with these requirements.
%
Unlike standard EFTs (but familiar from SCET), the action will contain inverse momentum scales, which nevertheless yields a local $S$-matrix as it must since this construction is equivalent to the manifestly local off-shell superspace description.

%%%%%%%%%%%%%%%%%%
\subsection{Constructing the Kinetic Term}
\label{subsec:chiral_kinetic}

We next want to build the kinetic term for the chiral superfield defined in \Eq{eq:chiralsuperfield}.
%
As we argue in the following, the unique kinetic term allowed by the criteria listed above is: 
%
\begin{equation}
\label{eq:VcompKinetic}
\bV_{\rm comp} =  \frac{i}{2}\bPhi^\dagger \frac{\Box}{\PP}  \bPhi\,.
\end{equation}
%
Despite the inverse momentum scale, this kinetic term is in fact local since $\Box$ is parametrically small compared to $\PP$, in keeping with the analysis of \Refs{Cohen:2016jzp, Cohen:2016dcl}.
%
Note that $\bV_{\rm comp}$ in \Eq{eq:VcompKinetic} is bosonic, real,%
%
\footnote{Strictly speaking, $\bV_{\rm comp}$ is only real up to total derivative terms appearing in the Lagrangian.}
%
and has mass dimension three, thereby following the rules outlined in the previous section.


Writing out the kinetic term in components, we have  
\begin{align}
\label{eq:SUSYSCETLagrangian}
\mathcal{L}  =\D \bar{\D} \bV_{\rm comp} \,\Big|_{0} 
& =\frac{i}{2}\,  \D \s\bar{\D} \bigg[ \bPhi^\dagger \frac{\Box}{\PP}  \bPhi \bigg] \bigg|_{0}   \notag\\[10pt]
& =  \bigg[- \,\bPhi^\dagger\s  \Box\s \bPhi +\frac{1}{2} \big( \bar{\D}\s  \bPhi^\dagger \big)\frac{i\s \Box}{\PP} \big(\D\s \bPhi \big)\bigg]\, \bigg|_{0}  \notag\\[10pt]
& = -\phi^* \Box \phi + i\s u^\dag \frac{\Box}{\PP} u\,,
\end{align}
%
which are the canonical  light-cone kinetic terms for the $\phi$ and $u$ fields.
%
In the second line, we made use of the product rule $\D\s(\mathbf{X\s Y}) = (\D\s \mathbf{X})\mathbf{Y} \pm \mathbf{X}(\D\s \mathbf{Y})$, where the sign depends on whether $\mathbf{X}$ is bosonic $(+)$ or fermionic $(-)$.
%
We also used the anti-commutation relation and chiral condition to write $\D \s\Dbar\s \bPhi^\dagger = -2\s i\, \PP\s \bPhi^\dagger$.

It is illustrative to explain in detail why \Eq{eq:VcompKinetic} is the unique kinetic term, since similar arguments will be applied in \Sec{sec:Gaugetheory}.
%
The kinetic term has to be bilinear in superfields, with one chiral and one anti-chiral field to make sure that $\D\s \bar{\D}\s \bV_{\rm comp}$ does not vanish.
%
Naively, the closest analog to the usual canonical K\"ahler potential would be $\bV_{\rm comp} = \bPhi^\dagger\s \bPhi$, but this is disqualified since it has mass dimension two instead of three.
%
Additionally, this term has the wrong RPI-III transformation since $\bV_{\rm comp} \to e^{-\rpiiii} \,\bV_{\rm comp}$ is required to balance $\D\s\bar{\D} \to e^\rpiiii \,\D\s\bar{\D}$ to obtain an invariant action, whereas $\bPhi^\dagger \s\bPhi$ is invariant.
%
We can compensate for this by including either $\tilde{\PP}$ or $1/\PP$, both of which have RPI-III charge $+1$, but only $1/\PP$ is invariant under RPI-I.\footnote{It is interesting that the action is forced to have inverse momentum scales by RPI.  For example, the local operator $\bV_{\rm comp} = \bPhi\, \Dbar\s \D\,  \bPhi^\dagger = -2\s i\, \bPhi\, \PP\s\bPhi^\dagger$ has mass dimension three but has RPI-III charge $-1$ instead of $+1$.}
%
Then to achieve the correct mass dimension, we can insert factors of the RPI invariant $\Box$.
%
Altogether, this yields $\bV_{\rm comp} = \frac{i}{2}\s\bPhi^\dagger\s \frac{\Box}{\PP} \s\bPhi$ as claimed, with the factor of $i$ needed to ensure that $\bV_{\rm comp}^\dagger = \bV_{\rm comp}$ and the $1/2$ for canonical normalization of the kinetic terms.

One might be concerned that starting from the required bilinear $\bPhi^\dagger\s \bPhi$, there could be additional independent terms one could write down using alternate derivative choices.
%
However, the space-time derivatives $\tilde{\PP}$, $\PP_\perp$, and $\PP_\perp^*$ have non-trivial RPI-I transformations, and are as such not useful for constructing the kinetic term, since $\bPhi$ is RPI-I invariant.
%
Said another way, while it is possible for $\bV_{\rm comp}$ to involve $\tilde{\PP}$, $\PP_\perp$, or $\PP_\perp^*$ directly, integration by parts can always be used to combine them into $\Box$.
%
Regarding the super-covariant derivatives, note that since they are fermionic and $\bPhi$ is bosonic, $\D$ and $\bar{\D}$ have to come in pairs.  Then we can use $\{ \D, \bar{\D}  \} = -2\s i\,\PP$, integration by parts, and the chirality conditions to convert them to space-time derivatives.
%
Note that the discussion in this paragraph only holds for bilinear terms, where integration by parts is particularly powerful, but it will not hold in general, see~\Sec{subsect:KahlerDiscussion}.
%

%%%%%%%%%%%%%%%%%%
\subsection{Verifying RPI-II}
\label{sec:revisitRPI2}

To verify that \Eq{eq:SUSYSCETLagrangian} satisfies RPI-II, we have to work directly in components.
%
Following \Sec{subsec:RPIii}, we assume that \Eq{eq:uNLRPIii} is the correct RPI-II transformation law.
%
The scalar kinetic term is manifestly RPI-II invariant.
%
We can check the fermion kinetic term by direct computation:
%
\begin{align}
\mathcal{L} \supset i \s u^\dag\s \frac{\Box}{\PP}\s u \RPIii & i \left( u^\dagger + \kappa^*_{\rm II} \frac{\PP_\perp}{\PP}\s   u \right) \left( \frac{\Box}{\PP} - \frac{\Box (\kappa_{\rm II}^*\, \PP_\perp + \kappa_{\rm II}\, \PP_\perp^*)}{\PP^2} \right) \left( u + \kappa_{\rm II}\, \frac{\PP_\perp^*}{\PP}\s u \right) \nonumber \\[8pt] 
&\quad= i \s u^\dagger\s \frac{\Box}{\PP}\s u + \mathcal{O}\big(\kappa_{\rm II}^2\big)\,,
 \label{eq:fermion_check_rpi_ii}
\end{align}
%
confirming that the full kinetic term for a collinear chiral superfield satisfies RPI-II.
%
Together with RPI-I, RPI-III, and collinear SUSY, this confirms that \Eq{eq:SUSYSCETLagrangian} describes a theory with full $\mathcal{N} = 1$ SUSY, albeit written in a language where Lorentz invariance and half of SUSY is obscured.

%%%%%%%%%%%%%%%%%%
\subsection{Where is the K\"ahler Potential?}
\label{subsect:KahlerDiscussion}

We argued above that these constructions lack a ``superpotential,'' which means that we will need the new technology to be introduced in \Ref{Cohen:2019gsc} to write down mass terms and Yukawa interactions.%
%
\footnote{The Wess-Zumino model was studied in \Ref{Cohen:2016dcl}, but only with the help of external currents.}
%
In a similar spirit, it is natural to wonder if it is possible to write down a non-trivial ``K\"ahler potential,'' which would allow us to investigate higher-order interactions.


As a warm up, consider making the replacement $\Box \to \Box + m^2$ in \Eq{eq:VcompKinetic}.
%
This yields the Lagrangian
%
\begin{align}
\label{eq:SUSYSCETLagrangian_with_mass}
\mathcal{L}  = \D\s \bar{\D}\s \bV_{\rm comp} \Big|_{0}  \stackrel{?}{\supset} \frac{i}{2}\s \D\s \bar{\D}\s \bigg[ \bPhi^\dagger\s \frac{(\Box+m^2)}{\PP}  \bPhi \bigg] \bigg|_{0} = -\phi^* \big(\Box + m^2\big) \phi + i\s u^\dag \frac{\Box+m^2}{\PP} u,
\end{align}
%
which naively looks like a theory with mass terms.
%
However, since the fermion is now in a massive representation of the Lorentz group, its corresponding RPI transformations are not given by \Eq{eq:uNLRPIii}.
%
Specifically, the RPI-II transformations of $u$ must now depend on $m$, and this spoils the RPI invariance of the conjectured Lagrangian given in \Eq{eq:SUSYSCETLagrangian_with_mass}.
%
This should not come as a surprise, since from the top-down perspective, a mass term in SUSY yields non-trivial $F$-term equations of motion, which are absent from the present construction.


In fact, most non-canonical choices of $\bV_{\rm comp}$ will violate RPI-II in some way.
%
As a concrete example, consider a massless theory (such that \Eq{eq:uNLRPIii} still holds) with the following class of higher-dimension operators
%
\begin{equation}
\bV_{\rm comp} \stackrel{?}{\supset} \frac{i}{\Lambda^{(n+m-2)}} \big(\bPhi^\dagger\big)^n \s\frac{ \Box}{\PP}\s \big(\bPhi\big)^m + \text{h.c.} \, ,
\end{equation}
%
where $n$ and $m$ are integers and $\Lambda$ has mass dimension 1.
%
This term is bosonic and real, has mass dimension 3 and RPI-III charge $+1$, and is RPI-I invariant: it is therefore a candidate for inclusion in $\bV_{\rm comp}$.
%
By explicit computation, though, one can check that it violates RPI-II.
%
In fact, apart from introducing additional factors of $(\Box/\Lambda^2)$ into \Eq{eq:VcompKinetic}, we have been unable to identify any non-canonical $\bV_{\rm comp}$ that preserves RPI-II while still respecting RPI-I, RPI-III, and collinear SUSY.


From the bottom-up perspective, this simply emphasizes the importance of RPI-II in enforcing Lorentz invariance.
%
From the top-down perspective, it underscores an interesting fact about K\"ahler potentials.
%
Even in theories with a vanishing superpotential, non-canonical K\"ahler potentials generate non-zero $F$-terms proportional to fermion bilinears:
%
\begin{equation}
F^i = \frac{1}{2}\, \Gamma^i_{jk}\, \chi^j\, \chi^k,
\end{equation}
%
where $\Gamma^i_{jk}$ is the Christoffel connection derived from the K\"ahler metric.
%
Since our constructions lack auxiliary fields, we cannot generate such a term (at least not with a linear realization of collinear SUSY).


In the companion paper, we introduce superfields whose lowest component is $\tilde{u}$ and whose highest component is $F$, making it possible to realize non-trivial K\"ahler potentials (and superpotentials) after imposing RPI-II.
%
For this paper, though, we have only provided the technology for writing the Lagrangian for massless free chiral multiplets.
%
To obtain non-trivial interactions, we have to turn to gauge theories.


%%%%%%%%%%%%%%%%%%
\section{Maiden Voyage:  Gauge Theories}
\label{sec:Gaugetheory}

Now that we have gained experience applying the EFT rules of collinear superspace to a free chiral multiplet, it is straightforward to explore the structure of gauge theories.
%
In this section, we explain how gauge invariance constrains operators in collinear superspace, starting from the simplest case of an Abelian gauge theory and then lifting to a non-Abelian gauge theory.
%
Obviously, the latter case requires introducing interactions, which provides a non-trivial check of the collinear superspace formalism.


As discussed in \Sec{sec:Superfields}, the familiar $\mathcal{N} =1$ method of organizing gauge degrees of freedom into a vector multiplet is not possible in collinear superspace.
%
That said, the physical polarizations of the gauge field and the gaugino can be packaged into a chiral superfield $\bPhiA$ whose kinetic term is given by \Eq{eq:SUSYSCETLagrangian}. 
%
In what follows, we demonstrate how a residual gauge symmetry, along with RPI, can be used to derive the rest of the gauge theory Lagrangian.

%%%%%%%%%%%%%%%%%%
\subsection{Abelian Gauge Theory}
We begin with the Abelian case.  
%
As discussed in \Sec{sec:ProjectGauge}, our construction is based on a complex light-cone scalar field $\alc$ that is built from the two propagating gauge degrees of freedom.
%
Under RPI-I, $\alc$ has non-trivial transformation properties, so in order to package $\alc$ into a superfield, it is necessary to enforce light-cone gauge in collinear superspace, where $\bar{n}\cdot A = 0$ and $n\cdot A$ is non-propagating and therefore integrated out.
%
In light-cone gauge, $\alc$ is inert under both RPI-I and RPI-III.

Since we have written the gauge modes suggestively as a complex scalar $\alc$, it is clear how to package it into a gauge chiral superfield (see~\emph{e.g.}~\cite{Leibbrandt:1987qv} for a review):
%
\begin{align}
\label{eq:PhiA}
\bPhiA =\alc^* - \sqrt{2} i \eta \lambda^\dagger + i \eta^\dagger \eta \, \PP \alc^* \qquad &\text{with} \qquad \bar{\D} \s\bPhiA  = 0 \, , \notag\\[6pt]
\bPhiA^\dagger = \alc - \sqrt{2} i \eta^\dagger \lambda - i \eta^\dagger \eta \, \PP \alc  \hspace{32pt} &\text{with} \qquad \D\s \bPhiA^\dagger  = 0\,.
\end{align}
%
In analogy to \Eq{eq:udecom_reverse}, we have defined the propagating gaugino as $\lambda \equiv \xi^\alpha \lambda_\alpha$, which is operationally an anti-commuting scalar.
%
Note that in \Eq{eq:PhiA}, the chiral superfield contains the conjugate fields $\alc^*$ and $\lambda^\dagger$, and vice verse for the anti-chiral field.
%
This unusual organization of the degrees of freedom arises because one has to add $+1/2$ units of helicity to go from the lowest to highest component of a chiral multiplet, $0 \to +1/2$ for \Eq{eq:chiralsuperfield} and $-1 \to -1/2$ for \Eq{eq:PhiA}.
%
One can also understand this by matching to the full $\mathcal{N} = 1$ theory (see \Refs{Cohen:2016jzp, Cohen:2016dcl} and further discussion in \Ref{Cohen:2019gsc}).


Even after enforcing light-cone gauge, there is a residual gauge transformation on the chiral gauge superfield.
%
As mentioned in \Sec{sec:Superfields}, this can be parametrized by $\RCA$, a superfield that is both chiral and real (and therefore anti-chiral): 
%
\begin{align}
&\bPhiA \gauge \bPhiA + \s\PP_\perp^*\s \RCA \,, 
\label{eq:residualgaugetrans}
\end{align}
%
In components this yields,
%
\begin{align}
&\alc\,\, \gauge \alc +  \s \PP_\perp\s \omega \, , \hspace{27pt} \quad \lambda_\alpha\gauge \lambda_\alpha \,,
\end{align}
Note that the gaugino $\lambda$ does not transform since this is an Abelian model.
%
The transformation of the gauge scalar $\alc$ is inferred by inserting the standard gauge transformation $A_\mu \to A_\mu +  \s \partial_\mu \omega$ into \Eq{eq:def_script_A}.
%
Crucially, the residual gauge transformation $\omega$ is consistent with light-cone gauge.
%
To see this, note that 
\begin{align}
\label{eq:fullloretnzgaugetrans}
 \bar{n} \cdot A \gauge  \bar{n} \cdot A +  \PP \s \omega = \bar{n} \cdot A,
\end{align}
where in the last step we used the fact that $\omega$ is the lowest component of $\RCA$ and therefore satisfies $\PP \omega = 0$.
%
Thus, the light-cone gauge condition $\bar{n} \cdot A = 0$ is maintained by the residual gauge transformations defined in \Eq{eq:residualgaugetrans}.


Plugging the gauge chiral superfield into the chiral kinetic term from \Eq{eq:SUSYSCETLagrangian}, the component Lagrangian takes the desired form:
\begin{align}
\mathcal{L}= \frac{i}{2} \D\s \bar{\D} \bigg[ \bPhiA^\dagger \frac{\Box}{\PP}  \bPhiA \bigg] \bigg|_0= -\alc^* \Box \alc +  i\s  \lambda^\dagger\s \frac{\Box}{\PP}\s  \lambda \,.
\label{eq:AbelianL}
\end{align}
%
It is straightforward to check that this superspace Lagrangian is gauge invariant,
\begin{align}
\label{eq:abelian_gauge_inv}
\mathcal{L} \gauge  \mathcal{L} +  \frac{i^2}{2} \bigg[ \big(\PP_\perp^*\s \bar{\D}\s  \RCA^\dagger\big)\frac{ \Box}{\PP}\s \big(\D\s \bPhiA\big) - \big(\bar{\D}\s \bPhiA^\dagger \big)\s \frac{ \Box}{\PP}\s \big(\PP_\perp\s \D \s\RCA\big) \bigg] \bigg|_0 + \,\, \mathcal{O}\big(\RCA^2\big) = \mathcal{L}\,,
\end{align}
since $\RCA$ is both chiral and anti-chiral.
%
Note that we have used the fact that $\PP \s \RCA = 0$ to remove any terms arising from the anti-commutator $\{\D, \bar{\D}\} = -2 \s i  \, \PP$.


%%%%%%%%%%%%%%%%%%
\subsection{RPI for the Abelian Theory}

As is clear from \Tab{table:RPI-components}, the components of $\bPhiA$ have the same RPI-I and RPI-III transformations as the matter chiral superfield (assuming light-cone gauge).
%
Therefore, the RPI-I and RPI-III invariance and uniqueness of \Eq{eq:AbelianL} follow from the arguments in \Sec{subsec:chiral_kinetic}.
%
However, $\bPhiA$ has non-trivial RPI-II transformations, so we must check that \Eq{eq:AbelianL} is consistent with this symmetry.
%
For the $\lambda$ kinetic term of \Eq{eq:AbelianL}, it is RPI-II invariant for the same reasons as for the $u$ kinetic term of \Eq{eq:fermion_check_rpi_ii}.
%
Checking RPI-II for the $\alc$ kinetic term requires a new argument.


As is the case with all non-covariant gauge choices, light-cone gauge obscures Lorentz invariance, which here manifests by studying the RPI-II transformations.
%
Note that $\alc$, $n\cdot A$, and $\bar{n} \cdot A$ transform under RPI analogous to $\PP_\perp$, $\tilde{\PP}$, and $\PP$. 
%
Under RPI-II, the light cone scalar transforms as $\alc \to \alc + \frac{\kappa_{\rm II}}{\sqrt{2}}\, n \cdot A$, and plugging this into \Eq{eq:AbelianL} yields an apparent violation of RPI-II.
%
The resolution comes from realizing that  $\bar{n} \cdot A$ transforms as $\bar{n} \cdot A \to \bar{n} \cdot A + \sqrt{2}\left(\kappa_{\rm II}\, \alc^* + \kappa_{\rm II}^*\, \alc \right)$.
%
Thus, it is unsurprising that fixing $\bar{n}\cdot A =0$ obscures RPI-II. 


To verify RPI-II, we need to restore the terms in our Lagrangian that depend on $\bar{n}\cdot A$.
%
From the top down, the Lagrangian can be derived by expanding the full kinetic term on the light cone.
%
From the bottom up, though, it is also possible to reconstruct the correct operator by only considering the properties of the effective theory.
%
In particular, there is a unique gauge artifact term that is linear in $\bar{n}\cdot A$, is RPI-III invariant, and transforms under RPI-I into something that still vanishes once light-cone gauge is enforced.
%
Adding this term to the Lagrangian, yields
%
\begin{align}
\mathcal{L} \,\,\,  \supset  \,\,\, - \alc^* \Box \alc + \frac{1}{2} \bar{n} \cdot A \Box (n \cdot A) \,,
\end{align} 
such that under an RPI-II transformation
\begin{align}
\mathcal{L} \RPIii & - \left( \alc^* + \frac{\kappa_{\rm II}^*}{\sqrt{2}} \s n \cdot A \right) \Box \left( \alc + \frac{\kappa_{\rm II}}{\sqrt{2}}\s n \cdot A \right) + \frac{1}{2}\left( \bar{n} \cdot A +\sqrt{2} \left( \kappa_{\rm II}\s \alc^* + \kappa_{\rm II}^*\s \alc \right)\right) \Box (n\cdot A) \nonumber\\ 
& = \mathcal{L} + \frac{\kappa_{\rm II}^*}{\sqrt{2}}\s \big( (n\cdot A \Box \alc)  - \alc \Box (n\cdot A)  \big) + \text{h.c} = \mathcal{L}\,,
\end{align}
where we have integrated by parts and set $\bar{n} \cdot A = 0$ to show the final equality holds.
%
This demonstrates that the collinear superspace Abelian gauge theory respects both RPI and gauge symmetry.


%%%%%%%%%%%%%%%%%%
\subsection{Gauge Transformations and Covariant Derivatives}
Now that we have shown how the Abelian theory can be expressed in collinear superspace, we can lift this to non-Abelian theory, which requires the introduction of interactions.
%
Each gauge field has a corresponding chiral multiplet $\bPhiA^{a}$ labeled by the group index $a$, with corresponding residual gauge transformations $\RCA^a$.
%
Furthermore, to write down gauge-invariant interactions, we need to covariantize \Eq{eq:AbelianL}.
%
To this end, we introduce a non-Abelian covariant derivative in superspace,
%
\begin{align}
\label{eq:gaugecoveraintderivatives}
\nabla_\perp \bPhialc \equiv \PP_\perp \bPhialc - \frac{i}{\sqrt{2}}  \, g \, \big[\bPhiA^\dagger, \bPhialc\big]  \, , \,\, \quad \quad \nabla_\perp^* \bPhialc \equiv  \PP_\perp^* \bPhialc - \frac{i}{\sqrt{2}}  \, g \, \big[\bPhiA , \bPhialc^\dagger \big] \,,
\end{align}
%
%
where we are assuming that both operators are acting on a field with the same charge.
%
Here, we are using the matrix notation $\bPhiA = T^a \s \bPhiA^a$,  where $T^a$ are the adjoint generators of the gauge group defined as $\big(T^a\big)_{bc} = -  i\s f^{abc}$.
%
In terms of the matrix components, \Eq{eq:gaugecoveraintderivatives} becomes
 \begin{align}
 \label{eq:nablaComp}
&\nabla^{* ab }_\perp = \PP_\perp^* \delta^{ab} + g f^{cab} \bPhiA^{c}  \, , \,\, \quad \quad \nabla^{ab}_\perp = \PP_\perp \delta^{ab} - g f^{cab} \bPhiA^{c  \dagger}.
\end{align} 
%
Notice that the lowest component of $\nabla_\perp$ is related to the ordinary gauge-covariant derivative $D_\mu$ as 
%
\begin{equation}
\nabla_\perp \big|_0 = \PP_\perp - i\s g\s T^a \alc^a = \xi\s \sigma^\mu\s \tilde{\xi}^\dagger \left(\partial_\mu - i\s g\s T^a\s A^a_\mu\right) = \xi\s \sigma^\mu\s \tilde{\xi}^\dagger D_\mu.
\end{equation}
% 
The fermionic component of $\nabla_\perp$ involves the gaugino. 

The gauge transformations are now given by
\begin{align}
\label{eq:NonAbelianPhiGaugeTrans}
\bPhialc &\gauge e^{i\s g\s \RCA} \big(\bPhiA + i\s \sqrt{2}\, \PP_\perp^* \big) e^{-i\s g\s \RCA}, 
\end{align}
where we write the residual gauge transformation parameter in matrix form $\RCA = T^a \s \RCA^a$.
%
In matrix components,  $\bPhi^a \rightarrow \bPhi^a + (\nabla_\perp^*)_{ab} \RCA_b$, so that \Eq{eq:NonAbelianPhiGaugeTrans} becomes
\begin{align}
\alc^a &\gauge \alc^a +  \PP_\perp \omega^a  + g f^{abc} \alc^b \omega^c \,, \\
 \lambda^a &\gauge \lambda^a + g f^{abc} \omega^b \lambda^c,
\end{align}
and similarly for the conjugate fields.
%
This verifies that \Eq{eq:NonAbelianPhiGaugeTrans} reproduces the expected non-Abelian gauge transformations of the light-cone-projected degrees of freedom.
% 
The gauge transformation of $\nabla_\perp$ follows from that of $\bPhi_\alc$, with
\begin{alignat}{2}
\nabla_\perp \bPhialc &&\gauge e^{i\s g\s \RCA} \Big(\nabla_\perp \bPhiA - \sqrt{2} \, g \,\PP_\perp \PP_\perp^*  \Big) e^{-i\s g\s \RCA} \,, \notag\\[5pt]
\nabla_\perp^* \bPhialc &&\gauge e^{i\s g\s \RCA} \Big(\nabla_\perp^* \bPhiA -\sqrt{2} \,g \,\PP_\perp^* \PP_\perp^* \Big) e^{-i\s g\s \RCA} \,,
\label{eq:nabla_def_nonAbelian}
\end{alignat}
in matrix notation. 



One complication with introducing $\nabla_\perp$ and $\nabla_\perp^*$ is related to RPI.
%
These objects are inert under RPI-III and transform under RPI-I as
%
\begin{equation}
\label{eq:nablaperp_rpii}
\nabla_\perp  \bPhialc \RPIi  \nabla_\perp  \bPhialc + \kappa_{\rm I}^*\, \PP  \bPhialc \,,
\end{equation}
%
and similarly for $\nabla_\perp^*$.
%
However, $\tilde{\PP}$ transforms under RPI-I as $\tilde{\PP} \to \tilde{\PP} + \rpii\, \PP_\perp+\kappa_{\rm I}^*\, \PP^*_\perp$, and the mismatch between $\PP_\perp$ and $\nabla_\perp$ makes it more complicated to verify RPI-I below.


Another way to view this mismatch is that, because $\bar{n}\cdot A$ and $n\cdot A$ are not present in light-cone gauge, there is no way to write gauge-covariant versions of $\PP$ and $\tilde{\PP}$. 
%
Importantly, $\nabla_\perp$ and $\nabla_\perp^*$ alone are sufficient for writing down gauge-invariant interactions in a pure gauge theory without matter.
%
With matter, a covariant version of $\tilde{\PP}$ is required, as discussed in \Sec{subsec:chargedmatter}, which will also help to make RPI-I manifest in the companion paper \cite{Cohen:2019gsc}.


%%%%%%%%%%%%%%%%%%
\subsection{Non-Abelian Gauge Theory} 

Using \Eq{eq:gaugecoveraintderivatives}, we can now write down a Lagrangian in superspace that is invariant under the residual non-Abelian gauge transformation $\RCA$.
%
Making the replacement in \Eq{eq:AbelianL}:
%
 \begin{equation}
 \label{eq:covariantBoxReplacement}
\frac{\Box}{\PP} = \tilde{\PP} - \frac{\PP_\perp \PP_\perp^*}{\PP} \quad \Longrightarrow \quad \tilde{\PP} - \nabla_\perp \frac{1}{\PP} \nabla^*_\perp \, ,
\end{equation} 
and introducing explicit group indices from \Eq{eq:nablaComp}, 
%
we have the proposed Lagrangian:   
 \begin{align}
 \label{eq:LnonAbelian}
 \mathcal{L} \,\, &= \,\, \frac{i}{2} \D \bar{\D} \Bigg[ \bPhiA^{a \dagger}  \left( \delta^{ac} \s \tilde{\PP}  - \s \nabla^{ab}_\perp \frac{1}{\PP} \nabla^{*bc}_\perp \right) \bPhiA^c \Bigg] \Bigg|_0  \,.
\end{align} 
%
We now will argue that \Eq{eq:LnonAbelian} is the unique dimension-four Lagrangian allowed by gauge invariance and RPI.
%
Dropping terms of order $\RCA^2$ and expanding out the covariant derivatives, the residual gauge transformations result in two classes of terms.
%
The first class of terms vanish following the same logic as \Eq{eq:abelian_gauge_inv}.
%
Schematically, these looks like
%
\begin{align}
\label{eq:gaugeTransOfTildePPterm}
\D \bar{\D} \bigg[ \bPhiA^{a} \, \hat{\mathcal{O}} \, \RCA^a  \bigg]  \Bigg|_0  =  \bigg[ ( \D \bPhiA^{a}) \, \hat{\mathcal{O}} \, (\bar{\D} \RCA^a)  \bigg]  \Bigg|_0 + \big(\text{terms} \propto \PP  \, \RCA^a \big) = 0 \,,
\end{align}
%
and similarly for the conjugate expression, where $\hat{\mathcal{O}}$ is some differential operator involving $\PP_\perp$, $\PP_\perp^*$, and $\tilde{\PP}$.
%
Since $\RCA$ is simultaneously chiral and anti-chiral, these vanishes under $\D\s \bar{\D}$.
%
Additionally, we have invoked the anti-commutation relation $\big\{\D, \bar{\D} \big\} \RCA= 2\s i\, \PP\s \RCA  = 0$.
%
The second class of terms contain products of fields like $\bPhi_{\alc}^{\dagger a} \, \bPhi_{\alc}^b \, \RCA^c$ with insertions of derivatives and an overall structure constant.
%
After using integration by parts and combining with Hermitian conjugates terms, these cancel among themselves due to the asymmetry of $f^{abc}$. 


To see why RPI-III holds, note that all of the terms in \Eq{eq:LnonAbelian} have the same RPI-III charges as their Abelian counterparts.
%
RPI-I is a bit more subtle to verify, for the reasons mentioned around \Eq{eq:nablaperp_rpii}.
%
It is convenient to write out the Lagrangian more explicitly:  
%
\begin{align}
\label{eq:nonablianLExpanded}
\mathcal{L}
&= \frac{i}{2}\D \bar{\D} \bigg[ \bPhiA^{a \dagger} \frac{\Box}{\PP} \bPhiA^a  \bigg] \bigg|_0 - \frac{i}{2}\s g\s f^{abc}\s \D\s \bar{\D} \bigg[  \left(\bPhiA^{\dagger a}\s \bPhiA^b\right) \frac{\PP_\perp^*}{\PP} \bPhiA^c - \bPhiA^{\dagger a}\s \frac{\PP_\perp}{\PP}\s \left(\bPhiA^{\dagger b}\s \bPhiA^c  \right) \bigg] \bigg|_0 \notag \\[5pt] 
& \qquad - \frac{i}{2} \s g^2\s f^{abc}\s f^{che}\s \D \s \bar{\D} \bigg[  \left(\bPhiA^{\dagger a}\s \bPhiA^b\right) \frac{1}{\PP} \left( \bPhiA^{\dagger h}\s \bPhiA^e \right) \bigg] \bigg|_0  \,\,.
\end{align}
%
The first and last terms in this expression are manifestly RPI-I invariant.
%
For the two middle terms, note that $\PP^*_\perp / \PP \to \PP^*_\perp / \PP + \rpii$, so under RPI-I, we have 
%
\begin{equation}
\label{eq:gauge_rpii_check}
\mathcal{L} \RPIi \mathcal{L}  - 2 \, i \, \rpii  \, \s g\s f^{abc}\s \D\s \bar{\D} \Big[ \bPhiA^{\dagger a}\s \bPhiA^b \bPhiA^c \Big]  + \text{h.c.} = \mathcal{L},
\end{equation}
%
where in the last step we have used the fact that $f^{abc}$ is completely antisymmetric.
%
This highlights the link between light-cone gauge invariance and RPI, which can be traced to the link between gauge redundancy and Lorentz invariance. 


Verifying RPI-II again requires going to components and checking the invariance explicitly.
%
Because this is a rather tedious exercise, here we appeal to the top-down construction in~\Refs{Cohen:2016jzp,Cohen:2016dcl}, which had to satisfy RPI-II since it was derived by matching to  the full Lorentz-invariant theory in light-cone gauge.
%
The expression in \Eq{eq:nonablianLExpanded} is identical to the Lagrangian derived in~\Refs{Cohen:2016jzp,Cohen:2016dcl} (up to conventions and superspace derivative manipulations), which implies that RPI-II is indeed satisfied.
%

%
Finally, to see why this term is unique, we can appeal to the same logic as in \Sec{subsec:chiral_kinetic}.
%
Apart from the replacement of $\PP_\perp$ with $\nabla_\perp$ (and the corresponding replacement of $\Box$ in \Eq{eq:covariantBoxReplacement}), there are no additional ingredients in the gauge case compared to the free chiral multiplet.
%
Therefore, without introducing any new mass scales, \Eq{eq:LnonAbelian} is the unique Lagrangian one can write consistent with gauge invariance and RPI.


%%%%%%%%%%%%%%%%%%
\subsection{Where is Charged Matter?}
\label{subsec:chargedmatter}

Armed with the covariant derivatives in \Eq{eq:gaugecoveraintderivatives}, one might naively think that it would be straightforward to add interactions involving charged matter.
%
For a charged matter chiral multiplet $\bM$, it should transform under the residual gauge transformation $\RCA$ as
%
\begin{align}
\bM  &\gauge  e^{i\s g\s \RCA} \bM,\\
\bM^\dagger   &\gauge   \bM^\dagger e^{-i\s g\s \RCA^\dagger} = \bM^\dagger e^{-i\s g\s \RCA},
\end{align}
%
where we have enforced the chirality/reality of $\RCA = \RCA^\dagger$. 
%
One can verify that the covariant derivative acts as expected,
%
\begin{align}
\nabla_\perp^* \bM  \gauge  e^{i\s g\s \RCA} \nabla_\perp^* \bM,
\end{align}
and one might be tempted to propose the candidate Lagrangian:
%
\begin{equation}
\label{eq:candidatematter}
\mathcal{L}_{\text{candidate}} \,\,\,  \stackrel{?}{\supset} \,\,\, \frac{i}{2} \D \bar{\D} \Bigl[ \bM^\dagger  \left( \tilde{\PP}  - \nabla_\perp \frac{1}{\PP} \nabla_\perp^* \right)  \bM \Bigr] \Big|_{0}  \,,
\end{equation}
in analogy with the non-Abelian gauge kinetic term in \Eq{eq:LnonAbelian}.


It is easy to check, however, that \Eq{eq:candidatematter} is neither gauge invariant nor RPI-I invariant.
%
Specifically, the manipulations below \Eq{eq:gaugeTransOfTildePPterm} and in \Eq{eq:gauge_rpii_check} no longer work, because arguments invoking the asymmetry of $f^{abc}$ fail when $\bM$ and $\bPhiA$ are distinct fields.
%
Because $\PP\s \RCA = 0$, we do not need a covariant version of $\PP$ to achieve gauge invariance, but we do need a covariant version of $\tilde{\PP}$ which involves $n \cdot A$.
%
This same $n \cdot A$ term is relevant for restoring RPI-I, but as discussed many times, this field does not appear in the present light-cone-gauge construction.


Another way to understand why this construction fails is that SUSY gauge theories with charged matter involve non-zero $D$-term auxiliary fields.
%
In the companion paper, we introduce a novel real superfield with non-trivial RPI transformation properties whose components include $n \cdot A$ and $D$~\cite{Cohen:2019gsc}.
%
In this way, the restoration of gauge invariance, RPI-I, and the required auxiliary fields are all achieved using related machinery.
%
From this perspective, the reason why $n \cdot A$ did not need to appear in the pure gauge Lagrangian in \Eq{eq:LnonAbelian} is that $D$ is identically zero in the full $\mathcal{N} = 1$ construction.
%
We leave a more detailed discussion of this point to the companion paper.


%%%%%%%%%%%%%%%%%%
\section{Future Horizons}
\label{sec:Outlook}
%%%%%%%%%%%%%%%%%%
In this paper, we provided a set of rules for constructing on-shell SUSY Lagrangians directly in collinear superspace, without any reference to the original Lorentz-invariant description.
%
This can be contrasted to the approach advocated in~\Refs{Cohen:2016jzp, Cohen:2016dcl}, where the Lagrangian was derived from the full $\mathcal{N} = 1$ theory by fixing a light cone and integrating out non-propagating degrees of freedom in superspace.
%
We now have a set of fully-consistent EFT rules for collinear superspace, based on the simple restriction given in \Eq{eq:tilde_eta_zero}, which yields a superspace where $\theta^2 = 0$.
%
This restriction selects a SUSY sub-algebra that is expressed on a light cone, and whose representations are built using only propagating degrees of freedom.
%
Furthermore, we were able to express the residual light-cone gauge invariance (encoded using the novel superfield $\RCA$), which was then used to derive the Lagrangian of both Abelian and non-Abelian gauge theories.
%
A formalism for reintroducing non-propagating degrees of freedom will be provided in \Ref{Cohen:2019gsc}, which is necessary to construct Wess-Zumino models and matter/gauge interactions where auxiliary $F$ and $D$ terms are essential.


While this was in some ways an academic exercise for $\mathcal{N}=1$ SUSY, which of course has a simple Lorentz-invariant superspace formulation using off-shell degrees of freedom, there are a number of aspects of our construction which are interesting in their own right.
%
We described RPI in the language of spinor/helicity, which is not so commonly encountered in the EFT literature. 
%
We introduced a superspace gauge-covariant derivative $\nabla_\perp$, which has no analog (to our knowledge) in the standard $\mathcal{N}=1$ treatment.
%
Beyond the novel real and chiral gauge parameter $\RCA$, even more exotic superfields will be encountered in \Ref{Cohen:2019gsc}, defined by mixed constraints involving both spacetime and superspace derivatives.


Ultimately, our hope is that these collinear superspace rules will generalize in a straightforward way to theories with $\mathcal{N} > 1$ SUSY (or even to theories with $d > 4$).
%
It is well known that the standard superspace approach only works for $\mathcal{N} = 1$, so discovering the underlying rules for an $\mathcal{N} > 1$ collinear superspace could in principle be useful to achieve a deeper understanding of these theories.
%
For example, perhaps the uniqueness of the $\mathcal{N}=4$ Lagrangian could be proven within collinear superspace directly, or maybe these constructions would illuminate the equivalence of the $\mathcal{N} = 3$ and $\mathcal{N} = 4$ Yang-Mills actions.
%
Additionally, it would be interesting to search for connections between the collinear superspace formalism and the on-shell recursive approach to scattering amplitudes, see \emph{e.g.}~\cite{Elvang:2013cua} for a review.
%
By obscuring Lorentz invariance, we hope this formalism will shed additional light on some of the amazing structures that emerge in SUSY field theories.

%%%%%%%%%%%%%%%%%%
\acknowledgments

We thank Martin Beneke for a helpful discussion about RPI.
%
TC is supported by the U.S. Department of Energy, under grant number DE-SC0011640 and DE-SC0018191, and a National Science Foundation LHC Theory Initiative Postdoctoral Fellowship, under grant number PHY-0969510.
%
GE is supported by the U.S. Department of Energy, under grant numbers DE-SC0011637 and DE-SC0018191, and a National Science Foundation LHC Theory Initiative Postdoctoral Fellowship, under grant number PHY-0969510.
%
JT thanks the Harvard Center for the Fundamental Laws of Nature for hospitality while this work was completed.
%
JT is supported by the Office of High Energy Physics of the U.S. Department of Energy under grant DE-SC-0012567 and by the Simons Foundation through a Simons Fellowship in Theoretical Physics.


\appendix
%%%%%%%%%%%%%%%%%%
\section{The Generators of RPI}
\label{app:RPIgen}
%%%%%%%%%%%%%%%%%%
In this appendix, we discuss the details of the RPI generators~\cite{Kogut:1969xa, Manohar:2002fd}.%
%
\footnote{See \Ref{Heinonen:2012km} for an analysis of Lorentz invariance and RPI generators in heavy particle effective theories.}
%
The Poincar\'e group is defined by
%
\begin{align}
\label{eq:poincare1}
\Big[P_\mu, P_\nu \Big] &= 0\,,\\[5pt] 
\label{eq:poincare2}
\Big[M^{\mu\nu},P^\rho \Big] &= i\s g^{\mu\rho}\,P^\nu-i\,g^{\nu\rho}\,P^\mu \,, \\[5pt]
\label{eq:poincare3}
\Big[M^{\mu \nu}, M^{\kappa \rho}\Big] &= - i g^{\mu \kappa} M^{\nu \rho} - i g^{\nu \rho} M^{\mu \kappa} + i g^{\mu \rho} M^{\nu \kappa} + i g^{\nu \kappa}M^{\mu \rho}\, ,
\end{align}
%
where $P_\mu = i\s  \partial_\mu$ is the generator of translations, and $M^{\mu \nu}$ is the usual anti-symmetric matrix of Poincar\'e generators
%
\begin{align}
& M^{\mu \nu} = 
\left[ \begin{array}{cccc}
0 & K^1 & K^2 & K^3  \\
-K^1 & 0 & - J^3 & J^2 \\
-K^2 & J^3 & 0 & -J^1 \\
-K^3 & - J^2 & J^1 & 0
  \end{array} \right],  \,
\end{align}
%
composed of rotations $M^{ij} = - \epsilon_{ijk}\s J^k$ and boosts $M^{0i} = K^i$, which satisfy the algebra
%
\begin{align}
\Big[J_i , J_j \Big] = i \s\epsilon_{ijk}\s J_k,\quad \Big[ J_i, K_j \Big] = i \s \epsilon_{ijk} \s K_k, \quad \text{and}\quad \Big[ K_i, K_j\Big] = - i\s \epsilon_{ijk} \s J_k\,. 
\end{align}
Projecting $M^{\mu\nu}$ onto the canonical frame $n^\mu = (1,0,0,1)$ and $\bar{n}^\mu = (1,0,0,-1)$ yields
\begin{align}
\label{eq:LCgenbreaking}
&R_{\rm I}^{\nu_\perp} = \bar{n}_\mu\s M^{\mu \nu_\perp} = M^{0 \nu_\perp} + M^{3 \nu_\perp} \,,\\[3pt] \nonumber
&R_{\rm II}^{\nu_\perp} = n_\mu\s M^{\mu \nu_\perp} = M^{0 \nu_\perp} - M^{3 \nu_\perp}\,, \\[3pt] \nonumber
& R_{\rm III} = n_\mu\s \bar{n}_\nu\s M^{\mu \nu} = 2\, M^{03} = 2\, K^3\,,
\end{align}
where $\nu_\perp = 1,2$, yielding five broken Lorentz generators.  Note that these projections can be equivalently expressed in terms of $\xi$ and $\tilde{\xi}$ using \Eq{eq:twospinorsn_alt}.




We can immediately identify RPI-III as the scalar operator corresponding to boosting along the light cone direction, $\hat{z}$ in the canonical frame.
%
The remaining four generators 
%
\begin{align}
R_{\rm I}^{1} = K^1- J^2 \, ,
& \quad R_{\rm I}^{2} = K^2 + J^1\,,\notag \\[3pt] 
R_{\rm II}^{1}  = K^1 + J^2 \, ,&\quad
R_{\rm II}^{2} = K^2 - J^1 \,,
\label{eq:RPI1And2}
\end{align}
%
correspond to boots and rotations about the directions transverse to the light cone.  



By inspecting \Eqs{eq:LCgenbreaking}{eq:RPI1And2}, we see that there is no explicit dependence on the $J_3$ generator.  
%
This is to be expected since we have picked the canonical frame, which points in the $\hat{z}$-direction, and the RPI transformations are the combinations of rotations and boosts which leave this direction unchanged.
%
However, when we compute the commutators
\begin{align}
\begin{array}{ll}
  \Big[ R^{\s\mu_\perp}_{\rm I}, R^{\nu_\perp}_{\rm I} \Big] = 0\,, &\qquad\qquad\qquad   \Big[ R^{\s\mu_\perp}_{\rm II}, R^{\nu_\perp}_{\rm II} \Big] = 0\,, \\[12pt]
 \Big[ R_{\rm I}^1, R_{\rm II}^1 \Big] =  i\s R_{\rm III}\,, &\qquad\qquad\qquad  \Big[ R_{\rm I}^2, R_{\rm II}^2 \Big] = i\s R_{\rm III}\,, \\[12pt]
  \Big[ R_{\rm I}^1, R_{\rm II}^2 \Big] =  -2\s i \s J_3\,, &\qquad\qquad\qquad  \Big[ R_{\rm I}^2, R_{\rm II}^2 \Big] = 2\s i\s J_3\,, \\[12pt]
  \Big[ R_{\rm I}^1, R_{\rm III} \Big] = -2\s i\s R_{\rm I}^1\,, &\qquad\qquad\qquad  \Big[ R_{\rm I}^2, R_{\rm III} \Big] = -2\s i\s R_{\rm I}^2\,, \\[12pt]
    \Big[ R_{\rm II}^1, R_{\rm III} \Big] = 2\s i\s R_{\rm II}^1\,, &\qquad\qquad\qquad  \Big[ R_{\rm II}^2, R_{\rm III} \Big] = 2\s i\s R_{\rm II}^2 \,,
\end{array}
  \label{eq:RPIComm}
\end{align}
we see $J_3$ is generated by successive RPI transformations.  
%
This is exactly the sense in which RPI secretly encodes the full Lorentz invariance, in that one can reconstruct the ``missing'' $J_3$ generator through the application of the RPI transformations alone.


\end{spacing}

%%%%%%%%%%%%%%%%%%
\section{The Rigging:  A Summary of Useful Formulae}
\label{app:UsefulFormulas}
%%%%%%%%%%%%%%%%%%

This appendix provides a set of reference formulas that are useful for deriving the results presented in the main text.  
%
Note that some expressions are redundant with the body of the paper, but we reproduce them here for convenience.



We work in Minkowski space with metric signature $g^{\mu \nu} = \textrm{diag}\left(+1,-1,-1,-1\right)$, and our $\gamma$-matrices are in the Weyl basis.
%
We follow spinor conventions of \Refs{Dreiner:2008tw, Binetruy:2006ad}. 
%
For a useful review of the conventions relevant for SUSY see pages 449--453 of \Ref{Binetruy:2006ad}. 


\subsection{Frame-Independent Expressions}

Here, we briefly summarize the frame-independent expressions.
%
For light-cone derivative we have:
%
\begin{align}
\bar{\sigma}^\mu \partial_\mu 
&= \tilde{\xi}^{\dagger \dot{\alpha}} \tilde{\xi}^\alpha\, \PP + \xi^{\dagger \dot{\alpha}} \xi^\alpha\, \tilde{\PP} + \xi^{\dagger \dot{\alpha}}  \tilde{\xi}^\alpha\, \PP_\perp + \tilde{\xi}^{\dagger \dot{\alpha}}  \xi^\alpha\, \PP^*_\perp \,, \notag\\[4pt]
\sigma^\mu \partial_\mu &=  \tilde{\xi}_\alpha \tilde{\xi}^\dagger_{\dot{\alpha}}\, \PP + \xi_\alpha \xi^\dagger_{\dot{\alpha}} \,\tilde{\PP} + \xi_\alpha \tilde{\xi}^\dagger_{\dot{\alpha}}\, \PP_\perp^* + \tilde{\xi}_\alpha \xi^\dagger_{\dot{\alpha}}\, \PP_\perp\, .
\label{eq:sigmapartialdecom}
\end{align}
Note that we can write $\Box = \PP \s \tilde{\PP} - \PP_\perp^* \PP_\perp$.
%
For light-cone spinors we have:
\bea
u_\alpha = \tilde{\xi}_\alpha u - \xi_\alpha \uu \qquad \text{so that} \qquad u = \xi^\alpha u_\alpha \,, \qquad \text{and} \qquad \tilde{\xi}^\alpha u_\alpha = \uu\,,
\label{eq:fermionDecomp}
\eea
where the choice of convention will be discussed further below.
%
Finally for the gauge field we have:
%
\begin{align}
\left(\sigma \cdot A \right)_{\alpha \dot{\alpha}} = \xi_\alpha\, \xi^{\dagger}_{\dot{\alpha}}\, n\cdot A + \tilde{\xi}_\alpha\, \tilde{\xi}^\dagger_{\dot{\alpha}}\, \bar{n}\cdot A+\sqrt{2} \,  \xi_\alpha\, \tilde{\xi}^\dagger_{\dot{\alpha}}\, \alc^* +\sqrt{2} \,  \tilde{\xi}_\alpha\, \xi^{\dagger}_{\dot{\alpha}}\, \alc\,.
\end{align}


\subsection{The Canonical Frame}
We often appeal to the \emph{canonical frame}, which is specified by 
%
\begin{align}
n^\mu = (1,0,0,1)\,, \qquad \bar{n}^\mu = (1,0,0,-1)\,,
\end{align}
%
and can be rewritten using a spinor helicity decomposition as
%
\begin{align}
n^\mu = \tilde{\xi}^\dag \bar{\sigma}^\mu \tilde{\xi} = \tilde{\xi} \sigma^\mu \tilde{\xi}^\dag \, \qquad \text{and} \qquad \bar{n}^\mu = \xi^\dag \bar{\sigma}^\mu \xi = \xi \sigma^\mu \xi^\dag\,.
\end{align}
%
This is equivalent to fixing the spinors to
\begin{align}
\label{eq:bosoniccanonicalspinors_again}
\xi^\alpha = (0,1)\,,\qquad \xi_\alpha = (-1,0)^{\intercal}, \qquad \tilde{\xi}^\alpha = (1,0)\,, \qquad \tilde{\xi}_\alpha = (0,1)^{\intercal}\,,
\end{align}
%
where
%
\begin{equation}
\xi^\alpha\s \tilde{\xi}_\alpha = 1, \qquad \tilde{\xi}^\alpha\s \xi_\alpha = - \epsilon^{\alpha \beta}\, \xi_\alpha \s\tilde{\xi}_\beta =-1\,, \qquad\xi^\dagger_{\dot{\alpha}}\s \tilde{\xi}^{\dagger \dot{\alpha}} = -1, \qquad \tilde{\xi}^\dag_{\dot{\alpha}}\, \xi^{\dag \dot{\alpha}} = 1\,.
\end{equation}

To express a Weyl spinor fermion in the canonical frame, we note that the projection operators that act as
%
\begin{align}
P_n u_\alpha = \frac{n\cdot\sigma}{2} \frac{\bar{n}\cdot \bar{\sigma}}{2}\, u_\alpha = u_2 \qquad \text{and} \qquad P_{\bar{n}} u_\alpha =\frac{\bar{n}\cdot\sigma}{2} \frac{n\cdot \bar{\sigma}}{2}\, u_\alpha =  u_1,
\end{align}
%
where we note that the $\alpha$ index is lowered in these expressions. 
%
We identify $u_2$ as the helicity aligned with the light-cone and $u_1$ is anti-aligned helicity, so that
\begin{align}
u \equiv u_2  \qquad \text{and} \qquad \tilde{u} \equiv u_1\,.
\end{align}
Given \eqref{eq:bosoniccanonicalspinors_again} this can be all be made consistent with the following conventional choice: 
\begin{align}
&u_\alpha =  \tilde{\xi}_\alpha u - \xi_\alpha \tilde{u}\,  \qquad \text{so that} \qquad \xi^\alpha u_\alpha = u \quad \text{and} \quad \tilde{\xi}^\alpha u_\alpha = \tilde{u}\,.
\end{align}
We note that this minus sign is not required for the expansion of $\theta^\alpha$ on the lightcone, and so we do not include it, see \eqref{eq:eta_def}.

A vector $V^\mu$ can be decomposed on the light cone as
%
\begin{align}
n\cdot V &= \tilde{\xi}^\dag \bar{\sigma}\cdot V \tilde{\xi} = \tilde{\xi} \sigma\cdot V \tilde{\xi}^\dag= V_0 + V_3 = -\big(V^0 + V^3\big)\,, \notag \\[5pt]
\bar{n} \cdot V &= \xi^\dag \bar{\sigma}\cdot V \xi = \xi \sigma\cdot V \xi^\dag= V_0 - V_3 = -\big(V^0 + V^3\big)\,,\notag\\[5pt]
V_\perp &= \tilde{\xi}^\dag \bar{\sigma}\cdot V \xi = \xi \sigma\cdot V \tilde{\xi}^\dag =V_1 + i\s V_2 = -\big(V^1 + i\s V^2\big)\,, \notag\\[5pt] 
V_\perp^{*} &= \xi^\dag \bar{\sigma}\cdot V \tilde{\xi} = \tilde{\xi} \sigma\cdot V \xi^\dag =V_1 - i\s V_2 = -\big(V^1 - i\s V^2\big)\,, 
\end{align}
where in the final steps we have fixed to the canonical frame.
%
This can be conveniently packaged as
\begin{align}
\sigma \cdot V = 
\left(\begin{matrix}
n\cdot V & -V_\perp^* \\
-V_\perp & \bar{n}\cdot V
\end{matrix}\right)
\qquad\text{and} \qquad
\bar{\sigma} \cdot V = 
\left(\begin{matrix}
\bar{n}\cdot V & V_\perp \\
V_\perp^* & n\cdot V
\end{matrix}\right)\,.
\end{align}




\subsection{Conventions in Soft-Collinear SUSY}

It is useful to keep in mind the spinor structure of objects in LCG used in the soft-collinear SUSY paper \cite{Cohen:2016dcl}.
%
For instance,
\begin{align}
& \sigma^\mu \partial_\mu = 
\left[ \begin{array}{cc}
\np & \sqrt{2}\,\partial^*  \\
\sqrt{2}\, \partial & \nbp  \end{array} \right]_{\alpha \dot{\alpha}},  \,
& \bar{\sigma}^\mu \partial_\mu = 
\left[ \begin{array}{cc}
\nbp & -\sqrt{2}\,\partial^*  \\
-\sqrt{2}\, \partial & \np  \end{array} \right]^{\dot{\alpha} \alpha}.
\end{align}
%
Note the $\sqrt{2}$ difference between the $\partial_\perp$ definitions and the $\PP_\perp$ definition used in this paper.
%
Similar expressions hold for other contractions such as $\sigma^\mu A_\mu$.
%
These expressions are independent of the choice of $n^\mu$ and $\bar{n}^\mu$ direction. 

Note that throughout we include the Lorentz contraction in the definitions of $\partial_\perp^2$, as this is convenient when working with LCG scalars:
\bea
\partial_\perp^2 \equiv \partial_\perp^\mu \partial_{\perp \mu}  = - \partial_1^2 - \partial_2^2  = - 2\, \partial\s \partial^* \, ,
\eea
where we have converted to LCG derivatives.
%
This is in contrast to some places in the literature which relate $\partial_\perp^2$ to the explicit component expression with the opposite sign.
%
In terms of this notation 
\bea
\Box = \partial^\mu\s \partial_\mu = \nbp \,\np + \partial_\perp^2 = \nbp \s\np - 2\, \partial\s \partial^* \,.
\eea

\begin{spacing}{1.1}
\addcontentsline{toc}{section}{\protect\numberline{}References}% 
\bibliography{OnShellSUSY}
\bibliographystyle{utphys}
\end{spacing}

\end{document}
